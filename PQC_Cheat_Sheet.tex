% Author: Mario Schiener, 2024
% LinkedIn: https://www.linkedin.com/in/mario-schiener/
% Github: https://github.com/marioschiener

% Compile this with LuaLaTeX!


\documentclass[11pt,english,a4paper, landscape]{scrartcl}

% !TeX root = ../PQC_Cheat_Sheet.tex

% Author: Mario Schiener, 2024
% LinkedIn: https://www.linkedin.com/in/mario-schiener/
% Github: https://github.com/marioschiener

% TODO This may include a few more imports than necessary and could be cleaned up a bit. That's just what happens when re-using prembles from other templates and documents.

\usepackage[utf8]{inputenc}
\usepackage[english,ngerman]{babel}
\usepackage{tgadventor} % font
\usepackage{fontspec}% requires XeLaTex or LuaLaTeX
\usepackage{fontawesome5}
\usepackage[table, dvipsnames]{xcolor}
\usepackage{tcolorbox}
\usepackage{svg}
\usepackage{wrapfig}
\usepackage{booktabs}
\usepackage{rotating}
\usepackage{graphicx}
\usepackage{times}
\usepackage{odsfile}
\usepackage{smartdiagram}
\usepackage{tikz}
\usetikzlibrary{matrix}
\usetikzlibrary{calc}
\usetikzlibrary{decorations.text,shadings, backgrounds, shapes.misc, positioning, arrows.meta, chains, shapes.symbols}	%for rounded rectangle ``buttons'', Speedometer diagram
\usepackage{siunitx}
\usepackage{tabularray}
\usepackage{tabularx}
\usepackage{multirow}
\usepackage{adjustbox}
\usepackage{enumitem}
\usepackage{makecell}
\usepackage{xifthen}
\usepackage{tikz}
\usepackage{lipsum}
\usepackage{float}
\usepackage{amssymb}
\usepackage{amsmath}
\usepackage{hyperref}
\usepackage[toc, acronym]{glossaries} % keep after hyperref
\usepackage{scrlayer-scrpage}
\usepackage[a4paper]{geometry}

% ==========================================================================================================
% COLORS, FONTS, LINKS
% ==========================================================================================================

% Colors
\definecolor{themeaccent}{RGB}{251,178,10}
\definecolor{themeblue}{RGB}{9,102,242}
\definecolor{themegreen}{RGB}{7,244,50}
\definecolor{themered}{RGB}{244,14,56}
\definecolor{themeyellow}{RGB}{255,204,7}
\definecolor{themeorange}{RGB}{251,151,10}
\definecolor{themewhite}{RGB}{220,220,220}	% Black!50
\definecolor{themegreylight}{RGB}{128,128,128}	% Black!50
\definecolor{themegreymedium}{RGB}{68,70,74}
\definecolor{themegreydark}{RGB}{64,64,64}	% Black!25
\definecolor{themegreyverydark}{RGB}{32,32,32}

% Fonts
\newfontfamily\montserrat{Montserrat-Alt1} % requires XeLaTex or LuaLaTeX
\newfontfamily\montserratlight{Montserrat-Alt1-ExtraLight} % requires XeLaTex or LuaLaTeX
\newfontfamily\montserratthin{Montserrat-Alt1-Thin} % requires XeLaTex or LuaLaTeX
\newfontfamily\montserratmedium{Montserrat-Alt1-Medium} % requires XeLaTex or LuaLaTeX
\newfontfamily\montserratsemibold{Montserrat-Alt1-SemiBold} % requires XeLaTex or LuaLaTeX
\newfontfamily\montserratbold{Montserrat-Alt1-Bold} % requires XeLaTex or LuaLaTeX

\setsansfont[Ligatures=TeX]{Tex Gyre Adventor}
\setmainfont[Ligatures=TeX]{Tex Gyre Adventor}
\renewcommand{\familydefault}{\sfdefault}


% TODO is this used in this document?
\newcommand{\highlight}[1]{\textit{\textcolor{themeaccent}{#1}}}

\parindent=0pt

% Links
\hypersetup{
    colorlinks=true,
    pdfborderstyle={/S/U/W 1},
    linkcolor=black,
    linkbordercolor=themeyellow,
    urlcolor=themeblue,
    urlbordercolor=themeaccent,
    citecolor=themegreylight
    }



\newcommand{\email}[2][]{\ifthenelse{\equal{#1}{}}{\href{mailto:#2}{\texttt{#2}}}{\href{mailto:#2}{\texttt{#1\,\,#2}}}}
\newcommand{\iconurl}[2][]{\ifthenelse{\equal{#1}{}}{\url{#2}}{\textcolor{themeaccent}{#1}\,\,\url{#2}}}
\newcommand{\iconhref}[3][]{\ifthenelse{\equal{#1}{}}{\href{#2}{#3}}{\href{#2}{#1 #3}}}

\newcommand{\tbd}{\textcolor{themeaccentsecondary}{TBD}}
\newcommand{\tbv}{\textcolor{themeaccentsecondary}{TBV}}

\DeclareMathOperator{\avg}{avg}


% ==========================================================================================================
% Vertically centered inline images
% ==========================================================================================================

\newcommand{\vcenteredincludesvg}[2]{\begingroup\setbox0=\hbox{\includesvg[#2]{#1}}\parbox{\wd0}{\box0}\endgroup}
\newcommand{\vcentered}[1]{\begingroup\setbox0=\hbox{#1}\parbox{\wd0}{\box0}\endgroup}


% ==========================================================================================================
% PAGE THEME& GEOMETRY
% ==========================================================================================================

\newcommand{\setpagetheme}[3]{

	\pagestyle{scrheadings}
	\clearpairofpagestyles
	%\clearscrheadfoot
	\automark[subsection]{section}


	% Margins and Background
	\newgeometry{left=1cm,right=1cm,top=2.4cm,bottom=0.75cm,headsep=0.75cm, footskip=2cm,includefoot}


	% Picture
	\AddToHook{shipout/background}{
		\begin{tikzpicture}[remember picture,overlay]
   				 \node[opacity=1] at (current page.center) {\includesvg{resources/Header}};
		\end{tikzpicture}
	}

	\addtokomafont{pagenumber}{\scshape\color{themegreylight}}

	%Header
	\ihead{\Huge \montserrat \textcolor{themeaccent}{#1}}

	%Footer
	\ofoot{\textcolor{themegreylight}{\pagemark}}
	\ifoot{\normalfont\scshape \textcolor{themegreylight}{#2}}
	\cfoot{\normalfont\scshape \textcolor{themegreylight}{#3}\\\textcolor{themegreylight}{\tiny No responsibility is taken for the correctness of this information.}}
}




% ==========================================================================================================
% ICONS
% ==========================================================================================================

% Note that for most icons, we will use some negative \hspace{- xyz} to shift folloing text closer to the icon. This is not done in the definition
% of the icons here on purpose as the right amount of shifting needed for nice looking results depends on the font size used in surrounding text or tabular environments.
% So, we do this right where we use the icons, even if it makes the code a bit ugly admittedly.

\newcommand{\icon}[4]
		\begin{tikzpicture}[scale=#4, transform shape]%
		\draw[draw=none,fill=#2] (0,0) circle (0.5);%
		\node[color=#3](icon) at (0,0) {#1};%
		\end{tikzpicture}
	}
}


\newcommand{\doubleicon}[5][themegreydark]
	\begin{tikzpicture}[scale=#5, transform shape]%
		\draw[draw=none,fill=#4] (0,0) circle (0.4);%
		\draw[draw=none,fill=themegreydark] (0.35,-0.2) circle (0.19);%
		\node[color=#1](icon) at (0,0) {\normalsize #2};%
		\node[color=themewhite,scale=0.9](icon) at (0.35,-0.2) {#3};%
	\end{tikzpicture}
	}
}

\newcommand{\tripleicon}[6][themegreydark]
	\begin{tikzpicture}[scale=#5, transform shape]%
		\draw[draw=none,fill=#4] (0,0) circle (0.4);%
		\draw[draw=none,fill=themegreydark] (0.35,-0.2) circle (0.19);%
		\draw[draw=none,fill=themegreydark] (0.35,0.2) circle (0.19);%
		\node[color=#1](icon) at (0,0) {\normalsize #2};%
		\node[color=themewhite](icon) at (0.35,-0.2) {#3};%
		\node[color=themewhite,scale=0.9](icon) at (0.35,0.2) {#6};%
	\end{tikzpicture}
	}
}

\newcommand{\tripleicongeneric}[6][themegreydark]
	\begin{tikzpicture}[scale=#5, transform shape]%
		\draw[draw=none,fill=#4] (0,0) circle (0.4);%
		\draw[draw=none,fill=themegreydark] (0.35,-0.2) circle (0.19);%
		\draw[color=themegreydark] (0.35,0.2) circle (0.19);%
		\node[color=#1](icon) at (0,0) {\normalsize #2};%
		\node[color=themewhite](icon) at (0.35,-0.2) {#3};%
		\node[color=themegreydark,scale=0.9](icon) at (0.35,0.2) {#6};%
	\end{tikzpicture}
	}
}

\newcommand{\lettericon}[4]
		\begin{tikzpicture}[scale=#4, transform shape]%
		\draw[draw=none,fill=#2] (-0.1,0) circle (0.125);%
		\node[color=#1](icon) at (-0.1,0) {\bfseries #3};%
		\end{tikzpicture}
	}
}


% ==========================================================================================================
% RANKINGS DIAGRAM
% ==========================================================================================================

\newcommand{\ranking}[4]{%
	\scalebox{#4}{
		\smartdiagramset{%
		set color list={themegreen,themeyellow,themeorange,themered,themered!75!black},
		sequence item border color=themegreydark,
		sequence item font size=\footnotesize,
		sequence item text color=themegreydark,
		sequence item width=2cm,
		sequence item height=1cm
		}
		#1\vcentered{\smartdiagram[sequence diagram]{#2}}#3
	}
}


% ==========================================================================================================
% BOXES
% ==========================================================================================================


\newenvironment{algorithmbox}[1]{%
		\begin{tcolorbox}[colback=themewhite,colframe=themegreydark,title=#1, coltitle=themeaccent, fonttitle=\scshape, coltext=themegreydark, halign=left]
	}%
	{%
	\end{tcolorbox}
}


% ==========================================================================================================
% Info Graphics
% ==========================================================================================================

% TODO maybe delete later, currently not used

\newlength{\heightsymbols}
\newlength{\heightbutton}
\newlength{\heighttoraise}

% Scale
% format: percentage, total number of symbols, symbol, font formatting, color
\newcommand{\absolutescale}[7][]{%
   	\pgfmathmultiply{#2}{#3}%
	\pgfmathdivide{\pgfmathresult}{100}%
	\pgfmathsetmacro{\bars}{int(round(\pgfmathresult))}
	\pgfmathadd{\pgfmathresult}{1}
	\pgfmathsetmacro{\barsnext}{int(round(\pgfmathresult)}


	%Some measurements for vertically centering everything
	\settoheight{\heightsymbols}{\hbox{#4}}
	\settoheight{\heightbutton}{\hbox{\custombutton{#5 30}{white}{white}}}
	\setlength{\heighttoraise}{\dimexpr(\heightbutton - \heightsymbols)/2\relax}

	{
	#5 % size
	\noindent\raisebox{\heighttoraise}{
		\foreach \j in {1,...,#3}{
			\ifnum\numexpr\j-1<#2\relax
				\textcolor{#6}{#4}\ignorespaces
			\else
				\textcolor{#7}{#4}\ignorespaces
			\fi
		}
	}\ifthenelse{\isempty{#1}}
	{}
	{\ignorespaces\,\,\custombutton{#5 #2/#3}{#6}{#7}}
	}
}


% Button
% format: content, fill color, text color

\newcommand{\custombutton}[3]{%
	\vcentered{%
		\begin{tikzpicture}
			\node [rounded rectangle, minimum width=1.25cm, minimum height=4mm, inner sep=2pt, fill=#2, text=#3, baseline=-0.5ex] {#1};
		\end{tikzpicture}
		}
}



% Speedometers

\pgfdeclarehorizontalshading{grad}{50bp}{
    color(0cm)=(themered);
	color(20bp)=(themeorange);
    color(30bp)=(themeyellow);
    color(50bp)=(themegreen)
}


% TODO maybe delete later, currently not used (only green variant below)

\def\RedAbsoluteSpeedometer(#1)[#2][#3]#4#5#6#7#8#9{%
	\begin{tikzpicture}
		\begin{scope}[shift={(#1)}]
			\edef\Mycount{0}%Variable to get the number of entries in the array
			\ifthenelse{\isempty{#4}}{\xdef\Mycount{#7}}{
				\foreach \element in {#4}{%For each element
					\pgfmathparse{int(\Mycount+1)}%increase in one unit the last value of \Mycount
					\xdef\Mycount{\pgfmathresult}%Refresh the  \Mycount value
				}%\Mycount get the value of the number of elements in given by #4 array input.
			}
			\pgfmathparse{180/\Mycount)}%Using the value of the number of inputs in the array
			\edef\Angle{\pgfmathresult}%Get the size of the angle sector for some range of valuis in the Gauje
			\pgfmathparse{180-(#6/#7)*180}
			\edef\ValueAngle{\pgfmathresult}


			% full half circle, goes in the background
			\def\leftpath{(180:#2)
					arc (180:0:#2)
					-- (0:#2*0.55)
					arc (0:180:#2*0.55)
					-- cycle;}
			\draw[draw=none,thick]\leftpath


			% segment to the right (> valueangle)
			\def\rightpath{(\ValueAngle:#2)
					arc (\ValueAngle:0:#2)
					-- (0:#2*0.55)
					arc (0:\ValueAngle:#2*0.55)
					-- cycle;}
			%\draw[draw=none,thick]\rightpath


			% full half circle with complete color scale
			\begin{scope}[on background layer]
				\shade[
					shading=grad,
					shading angle=180
					]\leftpath
			\end{scope}


			% cover the unused part through above layer
			\begin{scope}[on above layer]
				\shade[
					bottom color=themegreydark,
					top color=themegreydark,
					]\rightpath;
			\end{scope}


			\foreach \Stext/\Mtext [count=\i from 0] in {#4}{
				\def\Mypath{(180-\Angle*\i:#2)
					arc (180-\Angle*\i:180-\Angle*\i-\Angle:#2)
					-- (180-\Angle*\i-\Angle:#2*0.55)
					arc (180-\Angle*\i-\Angle:180-\Angle*\i:#2*0.55)
					-- cycle;}
				\draw[draw=none,thick]\Mypath

				% Numbers and text
				\path[decoration={text along path,text={| #8 |\Stext},text align=center},decorate]
					(180-\Angle*\i:#2*1.05) arc (180-\Angle*\i:180-\Angle*\i-\Angle:#2*1.05);


				\path[decoration={text along path,text={| #9 |\Mtext},text align=center, text color=white},decorate]
					(180-\Angle*\i:#2*#3) arc (180-\Angle*\i:180-\Angle*\i-\Angle:#2*#3);

			}

			\pgfmathparse{180-((180-\Angle)/#7*#6)}
			%\xdef\Value{\pgfmathresult} % by Nice cfr observation
			%\fill[themegreylight,rotate=\pgfmathresult] (90:#2*0.03)--(0:#2*0.75) -- (-90:#2*0.03)--cycle;
			%\fill [themegreylight](0,0) circle (#2*0.08);
			%\fill [themegreydark](0,0) circle (#2*0.03);

			% Caption
			\draw(-90:#2*0.15) node[font=\bfseries] {#5};
			\draw(-90:#2*-0.15) node[font=\bfseries] {\large \custombutton{#6}{themegreydark}{themewhite}};

		\end{scope}
	\end{tikzpicture}
}




\def\GreenAbsoluteSpeedometer(#1)[#2][#3]#4#5#6#7#8#9{%
	\begin{tikzpicture}
		\begin{scope}[shift={(#1)}]
			\edef\Mycount{0}%Variable to get the number of entries in the array
			\ifthenelse{\isempty{#4}}{\xdef\Mycount{#7}}{
				\foreach \element in {#4}{%For each element
					\pgfmathparse{int(\Mycount+1)}%increase in one unit the last value of \Mycount
					\xdef\Mycount{\pgfmathresult}%Refresh the  \Mycount value
				}%\Mycount get the value of the number of elements in given by #4 array input.
			}
			\pgfmathparse{180/\Mycount)}%Using the value of the number of inputs in the array
			\edef\Angle{\pgfmathresult}%Get the size of the angle sector for some range of valuis in the Gauje
			\pgfmathparse{180-(#6/#7)*180}
			\edef\ValueAngle{\pgfmathresult}


			% full half circle, goes in the background
			\def\leftpath{(180:#2)
					arc (180:0:#2)
					-- (0:#2*0.55)
					arc (0:180:#2*0.55)
					-- cycle;}
			\draw[draw=none,thick]\leftpath


			% segment to the right (> valueangle)
			\def\rightpath{(\ValueAngle:#2)
					arc (\ValueAngle:0:#2)
					-- (0:#2*0.55)
					arc (0:\ValueAngle:#2*0.55)
					-- cycle;}
			%\draw[draw=none,thick]\rightpath


			% full half circle with complete color scale
			\begin{scope}[on background layer]
				\shade[
					shading=grad
					]\leftpath
			\end{scope}


			% cover the unused part through above layer
			\begin{scope}[on above layer]
				\shade[
					bottom color=themegreydark,
					top color=themegreydark,
					]\rightpath;
			\end{scope}


			\foreach \Stext/\Mtext [count=\i from 0] in {#4}{
				\def\Mypath{(180-\Angle*\i:#2)
					arc (180-\Angle*\i:180-\Angle*\i-\Angle:#2)
					-- (180-\Angle*\i-\Angle:#2*0.55)
					arc (180-\Angle*\i-\Angle:180-\Angle*\i:#2*0.55)
					-- cycle;}
				\draw[draw=none,thick]\Mypath

				% Numbers and text
				\path[decoration={text along path,text={| #8 |\Stext},text align=center},decorate]
					(180-\Angle*\i:#2*1.05) arc (180-\Angle*\i:180-\Angle*\i-\Angle:#2*1.05);


				\path[decoration={text along path,text={| #9 |\Mtext},text align=center, text color=white},decorate]
					(180-\Angle*\i:#2*#3) arc (180-\Angle*\i:180-\Angle*\i-\Angle:#2*#3);

			}

			\pgfmathparse{180-((180-\Angle)/#7*#6)}
			%\xdef\Value{\pgfmathresult} % by Nice cfr observation
			%\fill[themegreylight,rotate=\pgfmathresult] (90:#2*0.03)--(0:#2*0.75) -- (-90:#2*0.03)--cycle;
			%\fill [themegreylight](0,0) circle (#2*0.08);
			%\fill [themegreydark](0,0) circle (#2*0.03);

			% Caption
			\draw(-90:#2*0.15) node[font=\bfseries] {#5};
			\draw(-90:#2*-0.15) node[font=\bfseries] {\large \custombutton{#6}{themegreydark}{themewhite}};

		\end{scope}
	\end{tikzpicture}
}





% Color palette: themewhite, themegreylight, themegreymedium, themegreydark, themegreen, themeblue, themered, themeyellow,themeorange
\colorlet{themeaccent}{themewhite}
\colorlet{themeaccentsecondary}{themeblue}

% ============================================================================================================
% Title,Date, Version, Document Status
% ============================================================================================================

\newcommand{\docstatus}{\textcolor{themered}{\scshape Draft}}
%\newcommand{\docstatus}{\textcolor{themeorangemedium}{\scshape Review}}
%\newcommand{\docstatus}{\textcolor{themewhite}{\scshape Released}}
%\newcommand{\docstatus}{\textcolor{themered}{\scshape Obsolete}}

\setpagetheme{%
Post-Quantum Cryptography Algorithm Cheat Sheets}{%	Title
\copyright\, Mario Schiener\quad%	Author details for footer
\href{https://www.linkedin.com/in/mario-schiener/}{\textcolor{themegreylight}{\faLinkedin}}\quad%
\href{https://github.com/marioschiener}{\textcolor{themegreylight}{\faGithub}}}%
{August 2024 -- v0.1}% Date, Version


% TODO Split algorithm ID cards into separate files
% TODO Implementation complexities and sizes
% TODO Classic McEliece, BIKE, HQC, FrodoKEM
% TODO XMSS metrics
% TODO Export the exact data behind the simplified values as CSV or something similar as auxiliary information


\begin{document}
	% ============================================================================================================
	% Explanations
	% ============================================================================================================
	\begin{minipage}[t]{0.4\textwidth} % left
		\vspace{-3mm}
		\begin{algorithmbox}{Symbols}
			\scriptsize
			{\bfseries  General color coding}\\[0.5\baselineskip]
			Best to Worst:
			\textcolor{themegreen}{$\blacksquare$}
			\textcolor{themeyellow}{$\blacksquare$}
			\textcolor{themeorange}{$\blacksquare$}
			\textcolor{themered}{$\blacksquare$}
			\textcolor{themered!65!black}{$\blacksquare$}\quad
			TBD: \textcolor{themeblue}{$\blacksquare$}\\[\baselineskip]

			{\bfseries  Symbols}
			\vspace{-0.5\baselineskip}
			\begin{tabbing}
				\textcolor{themegreydark}{\faExclamationCircle}\quad Recommendations differ depending on organization\\
				\=\textcolor{themegreydark}{\faSun[regular]}\quad Parameter set\hspace{3cm}
				\=\textcolor{themegreydark}{\faKey}\quad Key Generation\\
				\>\textcolor{themegreydark}{\faLock}\quad Encryption
				\>\textcolor{themegreydark}{\faUnlock}\quad Decryption\\
				\>\textcolor{themegreydark}{\faPen}\quad Signing
				\>\textcolor{themegreydark}{\faQuestionCircle}\quad Verification\\
				\>\textcolor{themegreydark}{\faCalendar}\quad Not yet standardized by NIST
				\>\textcolor{themegreydark}{\faMicrochip}\quad CPU Cycles\\
				\>\textcolor{themegreydark}{\faCode}\quad Implementation Code
				\>\hspace{-0.5mm}\lettericon{themewhite}{themegreydark}{C}{1}\hspace{-2mm} Implementation complexity\\
				\>\lettericon{themewhite}{themegreydark}{S}{1}\hspace{-1mm} Implementation size
			\end{tabbing}

			{\bfseries Security categories of parameter sets}\\[0.5\baselineskip]
			\doubleicon{\montserratbold V}{\faSun[regular]}{themegreen}{0.6}
			\doubleicon{\montserratbold IV}{\faSun[regular]}{themeyellow}{0.6}
			\doubleicon{\montserratbold III}{\faSun[regular]}{themeorange}{0.6}
			\doubleicon{\montserratbold II}{\faSun[regular]}{themered}{0.6}
			\doubleicon[themewhite]{\montserratbold I}{\faSun[regular]}{themered!65!black}{0.6} NIST Security Categories V, IV, III, II, I\\[\baselineskip]

			Higher means more secure.\\[\baselineskip]

			{\bfseries Implementation complexity and size}\\[0.5\baselineskip]
			\tripleicon{\montserratbold L}{\faCode}{themegreen}{0.6}{\bfseries C}
			\tripleicon{\montserratbold M}{\faCode}{themeyellow}{0.6}{\bfseries C}
			\tripleicon{\montserratbold H}{\faCode}{themered}{0.6}{\bfseries C} Low/Medium/High implementation complexity\\
			\tripleicon{\montserratbold L}{\faCode}{themegreen}{0.6}{\bfseries S}
			\tripleicon{\montserratbold M}{\faCode}{themeyellow}{0.6}{\bfseries S}
			\tripleicon{\montserratbold H}{\faCode}{themered}{0.6}{\bfseries S} Low/Medium/High implementation size\\[\baselineskip]
			Lower is better.\\[\baselineskip]

			{\bfseries Rating scales for parameter sizes and performance}\\[0.5\baselineskip]
			Best to Worst: \textcolor{themegreen}{$\blacksquare$} $n\leq2$, \textcolor{themeyellow}{$\blacksquare$} $n\{3,4\}$, \textcolor{themeorange}{$\blacksquare$} $n\in\{5,6\}$, \textcolor{themered}{$\blacksquare$} $n\in\{7,8\}$, \textcolor{themered!65!black}{$\blacksquare$} $n\geq9$
			\begin{tabbing}
				\=\tripleicon{\montserratbold n}{\faMicrochip}{themegreylight!50}{0.6}{\faKey}
				\=$\mathcal{O}(5^n)$ CPU kilo cycles for key generation\\
				\=\tripleicon{\montserratbold n}{\faMicrochip}{themegreylight!50}{0.6}{\faPen}
				\=$\mathcal{O}(5^n)$ CPU kilo cycles for signing\\
				\=\tripleicon{\montserratbold n}{\faMicrochip}{themegreylight!50}{0.6}{\faQuestionCircle}
				\=$\mathcal{O}(5^n)$ CPU kilo cycles for signature verification\\
				\=\tripleicon{\montserratbold n}{\faMicrochip}{themegreylight!50}{0.6}{\faLock}
				\=$\mathcal{O}(5^n)$ CPU kilo cycles for encryption / key encapsulation\\
				\=\tripleicon{\montserratbold n}{\faMicrochip}{themegreylight!50}{0.6}{\faUnlock}
				\=$\mathcal{O}(5^n)$ CPU kilo cycles for decryption / key decapsulation\\
				\>\doubleicon{\montserratbold n}{\faPen}{themegreylight!50}{0.6}
				\>$\mathcal{O}(2^n)$ KB of signature size\\
				\>\doubleicon{\montserratbold n}{\faLock}{themegreylight!50}{0.6}
				\> $\mathcal{O}(2^n)$ KB of ciphertext size\\
				\>\doubleicon{\montserratbold n}{\faKey}{themegreylight!50}{0.6}
				\> $\mathcal{O}(2^{(n-5)})$ KB of signature algorithm public key size\\
				\>\doubleicon{\montserratbold n}{\faKey}{themegreylight!50}{0.6}
				\> $\mathcal{O}(2^n)$ KB of encryption algorithm public key size\\
			\end{tabbing}
			\vspace{-6mm}
		\end{algorithmbox}
	\end{minipage}
	\hfill
	\begin{minipage}[t]{0.58\textwidth} % right
		\vspace{-3mm}
		\begin{algorithmbox}{How To Interpret This Cheat Sheet}
			\scriptsize
			{\bfseries The goal of this cheat sheet is to make it as easy as possible to figure out which algorithm to pick for a given use case. Algorithm ID cards break down algorithm parameter sets, their important values and performance characteristics. The cheat sheet is intended to help users primarily in technical roles, such as engineers, architects or software developers working with post-quantum cryptography.}\\[\baselineskip]

			The focus is to avoid giving specific numbers measured in bits, bytes or cycles as this makes makes comparing numbers across algorithms difficult. Instead, this complexity is simplified by only providing a
			{\bfseries color-coded number indicating the order of magnitude of each metric}.\\[\baselineskip]

			{\bfseries This approach prioritizes easy interpretation and comparability of metrics and in general quick informational gain over absolute precision of data -- remember this is a cheat sheet, not a standard! This document is not intended to replace the study of algorithm specifications. It just aims to point you in the right direction quickly.}\\[\baselineskip]

			The approach of focusing on orders of magnitude walks a fine line between treating too many things as ``equal'' and not simplifying things enough to be easy to read and compare. ``In the same order of magnitude'' usually refers to ``equal up to a factor of at most 10'', which is a very coarse way of comparing numbers. Treating metrics that differ by a factor of e.g. 9.9 as ``equal'' because 9.9 < 10 paints a distorted picture. In cryptography, factors of 5 or even 2 can make a significant difference in performance, both in theory and in practice. In order to still tease out the differences in metrics without throwing too many things together that actually differ significantly, this cheat sheet applies different scaling and ``orders of magnitude'' (i.e., not regarding base 10) for different metrics.\\[\baselineskip]


			It turns out that for {\bfseries metrics measured in (kilo) CPU cycle counts, i.e. algorithm performance, ``up to a factor of 5''} is a scale that is granular enough to work out the differences between algorithms while maintaining easy comparability. Those cycle counts heavily depend on the CPU used during measurement, hence the numbers need to be taken with a grain of salt, even if given exactly and not in terms of orders of magnitude.\\[\baselineskip]

			For {\bfseries signature and ciphertext sizes as well as key sizes, measuring numbers in kilobytes ``up to a factor of 2''} is well suited to work out the differences between algorithms while allowing for quick comparison. Specifically for signature public key sizes only, we offset the corresponding color coding by 5 orders of magnitude. This is because SLH-DSA has extremely small pubic keys compared to all other signature algorithms, which would extend the scale into negative numbers (e.g. for SLH-DSA-SHA2-128s, the public key has 32=2\textsuperscript{5} bytes, which corresponds to an order of magnitude of -5 when measuring in orders magnitude of factor 2 and in kilobytes). This phenomenon of algorithm metrics spanning a very large range of orders of (base 2) magnitudes does not occur to this extent for encryption algorithms, making an offset unnecessary.\\[\baselineskip]

			All values thus have a lower bound of 0. We do not limit the upper end of scales, but don't distinguish values greater than 10 anymore in terms of color coding. Please refer to the definitions on the left for symbol explanations, color coding and interpretation of numeric values.
			\vspace{0.7cm}
		\end{algorithmbox}
	\end{minipage}


	% ============================================================================================================
	% Main Content
	% ============================================================================================================

	\begin{minipage}[t]{0.7\textwidth}
		\vspace{0pt}
		% ============================================================================================================
		% Orders of Preference
		% ============================================================================================================
		\begin{algorithmbox}{Algorithm Choice Guidance: Rules Of Thumb}
			{\scriptsize The following orders of preference should be considered as general rules of thumb applicable to most use cases. They should however not be considered to be universal truths. Unstandardized algorithms are listed for the sake of completeness.\\}
			% ============================================================================================================
			% Signature Algorithms
			% ============================================================================================================
			\begin{minipage}[t]{0.5\textwidth}
				\vspace{0pt}
				{\scshape \faPen\, Signature Algorithms}\\
				{\scriptsize \bfseries Objective: Best All-Around Package \tbv}\\[0.1\baselineskip]
				\ranking{}{%
				\bfseries{ML-DSA},\bfseries{Falcon} \faCalendar,\bfseries{SLH-DSA},\bfseries{XMSS/LMS}*,\textcolor{themewhite}{\bfseries{New Candidates**} \faCalendar}}{}{0.6}\\[0.75\baselineskip]
				{\scriptsize \bfseries Objective: Performance of Key Generation \tbv}\\[0.1\baselineskip]
				\ranking{}{\bfseries{ML-DSA},\bfseries{SLH-DSA\\f variants},\bfseries{SLH-DSA\\s variants},\bfseries{Falcon} \faCalendar}{}{0.6}\\[0.75\baselineskip]
				{\scriptsize \bfseries Objective: Performance of Signing \tbv}\\[0.1\baselineskip]
				\ranking{}{\bfseries{ML-DSA},\bfseries{Falcon} \faCalendar, \bfseries{SLH-DSA\\f variants},\bfseries{SLH-DSA\\s variants}}{}{0.6}\\[0.75\baselineskip]
				{\scriptsize \bfseries Objective: Performance of Verification \tbv}\\[0.1\baselineskip]
				\ranking{}{\bfseries{Falcon} \faCalendar,\bfseries{ML-DSA},\bfseries{SLH-DSA\\s variants},\bfseries{SLH-DSA\\f variants}}{}{0.6}\\[0.75\baselineskip]
				{\scriptsize \bfseries Objective: Small Signatures \tbv}\\[0.1\baselineskip]
				\ranking{}{\bfseries{Falcon} \faCalendar,\bfseries{ML-DSA},\bfseries{SLH-DSA\\s variants},\bfseries{SLH-DSA\\f variants}}{}{0.6}\\[0.75\baselineskip]
				{\scriptsize \bfseries Objective: Small Public Keys \tbv}\\[0.1\baselineskip]
				\ranking{}{\bfseries{SLH-DSA},\bfseries{Falcon} \faCalendar,\bfseries{ML-DSA}}{}{0.6}\\[0.75\baselineskip]
				{\tiny * \,\,XMSS/LMS is not suitable for general purpose signatures}\\[-0.5\baselineskip]
				{\tiny ** Security still unclear, none recommended yet at the moment}
			\end{minipage}
			\hfill
			% ============================================================================================================
			% Page 1: Encryption/KEM Algorithms
			% ============================================================================================================
			\begin{minipage}[t]{0.5\textwidth}
				\vspace{0pt}
				{\scshape \faLock\, Encryption/Key Encapsulation Algorithms}\\
				{\scriptsize \bfseries Objective: Best All-Around Package \tbv}\\[0.1\baselineskip]
				\ranking{}{\bfseries{ML-KEM},\bfseries{HQC} \faCalendar, {\bfseries{C. McEliece} \faCalendar\,\faExclamationCircle}, \bfseries{BIKE} \faCalendar, \textcolor{themewhite}{\bfseries{FrodoKEM} \faExclamationCircle}}{}{0.6}\\[0.75\baselineskip]
				{\scriptsize \bfseries Objective: Performance of Key Generation \tbv}\\[0.1\baselineskip]
				\ranking{}{\bfseries{ML-KEM},\bfseries{HQC} \faCalendar, \bfseries{BIKE} \faCalendar, \bfseries{FrodoKEM} \faExclamationCircle, {\textcolor{themewhite}{\bfseries{C. McEliece} \faCalendar\,\faExclamationCircle}}}{}{0.6}\\[0.75\baselineskip]
				{\scriptsize \bfseries Objective: Performance of Encryption \tbv}\\[0.1\baselineskip]
				\ranking{}{{\bfseries{C. McEliece} \faCalendar\,\faExclamationCircle},\bfseries{ML-KEM},\bfseries{BIKE} \faCalendar,  \bfseries{HQC} \faCalendar, \textcolor{themewhite}{\bfseries{FrodoKEM} \faExclamationCircle}}{}{0.6}\\[0.75\baselineskip]
				{\scriptsize \bfseries Objective: Performance of Decryption \tbv}\\[0.1\baselineskip]
				\ranking{}{\bfseries{ML-KEM}, {\bfseries{C. McEliece} \faCalendar\,\faExclamationCircle},\bfseries{HQC} \faCalendar, \bfseries{FrodoKEM} \faExclamationCircle, \textcolor{themewhite}{\bfseries{BIKE} \faCalendar}}{}{0.6}\\[0.75\baselineskip]
				{\scriptsize \bfseries Objective: Small Ciphertext \tbv}\\[0.1\baselineskip]
				\ranking{}{{\bfseries{C. McEliece} \faCalendar\,\faExclamationCircle}, \bfseries{ML-KEM},\bfseries{BIKE} \faCalendar, \bfseries{HQC} \faCalendar, \textcolor{themewhite}{\bfseries{FrodoKEM} \faExclamationCircle}}{}{0.6}\\[0.75\baselineskip]
				{\scriptsize \bfseries Objective: Small Public Keys \tbv}\\[0.1\baselineskip]
				\ranking{}{\bfseries{ML-KEM},\bfseries{BIKE} \faCalendar, \bfseries{HQC} \faCalendar, \bfseries{FrodoKEM} \faExclamationCircle, {\textcolor{themewhite}{\bfseries{C. McEliece} \faCalendar\,\faExclamationCircle}}}{}{0.6}\\[0.75\baselineskip]
			\end{minipage}
		\end{algorithmbox}
		\begin{algorithmbox}{Pure PQC vs PQ/T Hybrid}
			\scriptsize
			This topic depends on too many factors (e.g. cost of migration, security considerations, risk profile, GRC requirements) to give general advice. {\bfseries For PQ/T hybrids, consider  ECC (e.g. secp256r1, Curve25519)} over RSA for the traditional component.
			\vspace{-1mm}
		\end{algorithmbox}
	\end{minipage}
	\begin{minipage}[t]{0.28\textwidth}
		\vspace{0pt}
		\begin{algorithmbox}{Security Category Choices}
			\scriptsize
			\begin{itemize}[leftmargin=*]
				\setlength\itemsep{0em}
				\item First, consider using \doubleicon{\montserratbold III}{\faSun[regular]}{themeorange}{0.6}\hspace{-3mm} as a baseline.\\
				\item Use \doubleicon{\montserratbold IV}{\faSun[regular]}{themeyellow}{0.6}\hspace{-3mm} or
				\doubleicon{\montserratbold V}{\faSun[regular]}{themegreen}{0.6}\hspace{-3mm} for more security if possible (i.e., if a decrease in performance is not a concern and if no constraints apply).\\
				\item Use \doubleicon[themewhite]{\montserratbold I}{\faSun[regular]}{themered!65!black}{0.6}\hspace{-3mm} or
				\doubleicon{\montserratbold II}{\faSun[regular]}{themered}{0.6}\hspace{-3mm} {\bfseries if and only if} \doubleicon{\montserratbold III}{\faSun[regular]}{themeorange}{0.6}\hspace{-3mm} or higher is not an option due to constraints (e.g. performance, memory, etc.).\\[\baselineskip]
			\end{itemize}
		\end{algorithmbox}
		\begin{algorithmbox}{Pure vs. Pre-Hashing}
			\scriptsize
			%\vspace{-2mm}
			\begin{itemize}[leftmargin=*]
				\setlength\itemsep{0em}
				\item {\bfseries First, consider using  pure} (i.e., without pre-hashing) as this is the general recommendation.\\
				\item {\bfseries Pre-Hashing may be considered if one or more of the following applies:}
				\begin{itemize}[leftmargin=*]
					\setlength\itemsep{0em}
					\item The message $M$ is too large to be sent to cryptographic module (CM) for hashing without significantly impacting performance. This may be the case e.g. in CMS related use cases such as S/MIME or code signing, or in cases of very narrow communication channels to the CM (e.g. between APDUs exchanged between smartcard and smartcard reader).
					\item The hash needs to be signed with different algorithms and would be computed repeatedly without pre-hashing.
					\item The specific hash function is not supported in a CM.
				\end{itemize}
			\end{itemize}
			\vspace{0.75mm}
		\end{algorithmbox}
	\end{minipage}




	% ============================================================================================================
	% ML-DSA
	% ============================================================================================================
	\newpage
	{\large \scshape \faPen\, Signature Algorithm Overview \& ID Cards}\\

	\begin{algorithmbox}{ML-DSA (Module-Lattice-Based Digital Signature Algorithm)}
		\begin{minipage}[t]{0.35\textwidth}
			\scriptsize
			\begin{center}
				\adjustbox{valign=t}{%
					\GreenAbsoluteSpeedometer(0,-10)[2.25cm][0.75]{Unusable/1-3,Poor/4-5,Good/6-8,Perfect/9-10}{Algorithm Overall Usability Score}{6.55}{10}{\tiny}{\tiny \montserratsemibold}\hspace{1.5cm}}
			\end{center}

			\begin{tabbing}
				\bfseries \scshape Previous Name\hspace{2.5cm}\=CRYSTALS-DILITHIUM\\
				\bfseries \scshape Specification\>\iconhref[\faLink]{https://csrc.nist.gov/pubs/fips/204/final}{FIPS 204}\\
				\bfseries \scshape Type\>Signature\\
				\bfseries \scshape Family\>Lattice\\
				\bfseries \scshape Standardization Status\>Standardized\\
				\bfseries \scshape Recommended By\>NIST, BSI, ANSSI\\
				\bfseries \scshape Hashing\>Pure, Pre-Hashing\\
				\bfseries \scshape Naming\>by $k\times l$ matrix $A$\\
				\>(e.g. $6\times5\,\rightarrow$ ML-DSA-65)
			\end{tabbing}
			\begin{tabular}[t]{l c  c  c}
				\scshape\bfseries Algorithm Implementation &\textcolor{themegreydark}{\faKey}&\textcolor{themegreydark}{\faPen}&\textcolor{themegreydark}{\faQuestionCircle}\\
				&&&\\
				\hline\\

				\scshape Complexity
				&\hspace{3mm}\tripleicon[themewhite]{\montserratbold ?}{\faCode}{themeaccentsecondary}{0.6}{\bfseries C}
				&\hspace{3mm}\tripleicon[themewhite]{\montserratbold ?}{\faCode}{themeaccentsecondary}{0.6}{\bfseries C}
				&\hspace{3mm}\tripleicon[themewhite]{\montserratbold ?}{\faCode}{themeaccentsecondary}{0.6}{\bfseries C}\\[2mm]
				\scshape Size
				&\hspace{3mm}\tripleicon[themewhite]{\montserratbold ?}{\faCode}{themeaccentsecondary}{0.6}{\bfseries S}
				&\hspace{3mm}\tripleicon[themewhite]{\montserratbold ?}{\faCode}{themeaccentsecondary}{0.6}{\bfseries S}
				&\hspace{3mm}\tripleicon[themewhite]{\montserratbold ?}{\faCode}{themeaccentsecondary}{0.6}{\bfseries S}
			\end{tabular}\\[1.5\baselineskip]
		\end{minipage}
		\hfill
		\begin{minipage}[t]{0.64\textwidth}
			\scshape \scriptsize
			\begin{tabular}[t]{c c  c  c  c  c c}
				\bfseries \makecell{Version\\{}} &  \bfseries \makecell{OID\\{}} &\bfseries \makecell{Security\\Category} & \bfseries \makecell{Performance\\{\faKey\,\quad\quad\faPen\,\quad\quad\faQuestionCircle}} &  \bfseries \makecell{Signature\\Size} & \bfseries \makecell{Public Key\\Size} & \bfseries \makecell{Suitable\\Pre-Hashing} \\
				&&&&&&\\
				\hline\\


				ML-DSA-44
				& 2.16.840.1.101.3.4.3.17
				& \hspace{3mm}\doubleicon{\montserratbold II}{\faSun[regular]}{themered}{0.6}
				& \hspace{3mm}\tripleicon{\montserratbold 3}{\faMicrochip}{themeyellow}{0.6}{\faKey}
				\tripleicon{\montserratbold 3}{\faMicrochip}{themeyellow}{0.6}{\faPen}
				\tripleicon{\montserratbold 2}{\faMicrochip}{themegreen}{0.6}{\faQuestionCircle}
				& \hspace{3mm}\doubleicon{\montserratbold 1}{\faPen}{themegreen}{0.6}
				& \hspace{3mm}\doubleicon{\montserratbold 5}{\faKey}{themeorange}{0.6}
				& SHA-256, SHA3-256\\

				ML-DSA-65
				& 2.16.840.1.101.3.4.3.18
				& \hspace{3mm}\doubleicon{\montserratbold III}{\faSun[regular]}{themeyellow}{0.6}
				& \hspace{3mm}\tripleicon{\montserratbold 3}{\faMicrochip}{themeyellow}{0.6}{\faKey}
				\tripleicon{\montserratbold 3}{\faMicrochip}{themeyellow}{0.6}{\faPen}
				\tripleicon{\montserratbold 3}{\faMicrochip}{themeyellow}{0.6}{\faQuestionCircle}
				& \hspace{3mm}\doubleicon{\montserratbold 1}{\faPen}{themegreen}{0.6}
				& \hspace{3mm}\doubleicon{\montserratbold 5}{\faKey}{themeorange}{0.6}
				& SHA-384, SHA3-384\\

				ML-DSA-87
				& 2.16.840.1.101.3.4.3.19
				& \hspace{3mm}\doubleicon{\montserratbold V}{\faSun[regular]}{themegreen}{0.6}
				& \hspace{3mm}\tripleicon{\montserratbold 4}{\faMicrochip}{themeyellow}{0.6}{\faKey}
				\tripleicon{\montserratbold 4}{\faMicrochip}{themeyellow}{0.6}{\faPen}
				\tripleicon{\montserratbold 3}{\faMicrochip}{themeyellow}{0.6}{\faQuestionCircle}
				& \hspace{3mm}\doubleicon{\montserratbold 2}{\faPen}{themegreen}{0.6}
				& \hspace{3mm}\doubleicon{\montserratbold 6}{\faKey}{themeorange}{0.6}
				& SHA-512, SHA3-512
			\end{tabular}
			\vfill
		\end{minipage}\\[\baselineskip]
		\hrule
		\vspace{1\baselineskip}
		\begin{minipage}[t]{0.49\textwidth}
			\scriptsize\faThumbsUp\, {\bfseries / Pros / Use If:}
			\begin{itemize}[leftmargin=*]
				\setlength\itemsep{0em}
				\item You need a general-purpose signature algorithm with decent specs in all categories
			\end{itemize}
		\end{minipage}
		\hfill
		\begin{minipage}[t]{0.49\textwidth}
			\scriptsize \faThumbsDown\, {\bfseries / Cons / Don't Use If:}
			\begin{itemize}[leftmargin=*]
				\setlength\itemsep{0em}
				\item You don't want a lattice-based algorithm
			\end{itemize}
		\end{minipage}\\[\baselineskip]
		{\tiny  {\bfseries \scshape Note:} Cycle counts for key generation, signing and verification depend on the CPU used. Values may vary on different CPUs and thus only a rough indicator.}
	\end{algorithmbox}


	% ============================================================================================================
	% Falcon
	% ============================================================================================================
	\begin{algorithmbox}{Falcon (Fast-Fourier Lattice-Based Compact Signatures over NTRU)}
		\begin{minipage}[t]{0.35\textwidth}
			\scriptsize
			\begin{center}
				\adjustbox{valign=t}{%
					\GreenAbsoluteSpeedometer(0,-10)[2.25cm][0.75]{Unusable/1-3,Poor/4-5,Good/6-8,Perfect/9-10}{Algorithm Overall Usability Score}{5.73}{10}{\tiny}{\tiny \montserratsemibold}\hspace{1.5cm}}
			\end{center}

			\begin{tabbing}
				\bfseries \scshape Previous Name\hspace{2.5cm}\=Falcon\\
				\bfseries \scshape Specification\>\iconhref[\faLink]{https://falcon-sign.info/}{Project Page}\\
				\bfseries \scshape Type\>Signature\\
				\bfseries \scshape Family\>Lattice\\
				\bfseries \scshape Standardization Status\>\faCalendar\, Pending\\
				\bfseries \scshape Recommended By\>\tbd\\
				\bfseries \scshape Hashing\>\tbd\\
				\bfseries \scshape Naming\>\tbd
			\end{tabbing}
			\begin{tabular}[t]{l c  c  c}
				\scshape\bfseries Algorithm Implementation &\textcolor{themegreydark}{\faKey}&\textcolor{themegreydark}{\faPen}&\textcolor{themegreydark}{\faQuestionCircle}\\
				&&&\\
				\hline\\


				\scshape Complexity
				&\hspace{3mm}\tripleicon{\montserratbold H}{\faCode}{themered}{0.6}{\bfseries C}
				&\hspace{3mm}\tripleicon{\montserratbold H}{\faCode}{themered}{0.6}{\bfseries C}
				&\hspace{3mm}\tripleicon[themewhite]{\montserratbold ?}{\faCode}{themeaccentsecondary}{0.6}{\bfseries C}\\[2mm]
				\scshape Size
				&\hspace{3mm}\tripleicon[themewhite]{\montserratbold ?}{\faCode}{themeaccentsecondary}{0.6}{\bfseries S}
				&\hspace{3mm}\tripleicon[themewhite]{\montserratbold ?}{\faCode}{themeaccentsecondary}{0.6}{\bfseries S}
				&\hspace{3mm}\tripleicon[themewhite]{\montserratbold ?}{\faCode}{themeaccentsecondary}{0.6}{\bfseries S}\\
			\end{tabular}\\[1.5\baselineskip]
		\end{minipage}
		\hfill
		\begin{minipage}[t]{0.64\textwidth}
			\scshape \scriptsize
			\begin{tabular}[t]{c c  c  c  c  c c}
				\bfseries \makecell{Version\\{}} &  \bfseries \makecell{OID\\{}} &\bfseries \makecell{Security\\Category} & \bfseries \makecell{Performance\\{\faKey\,\quad\quad\faPen\,\quad\quad\faQuestionCircle}} &  \bfseries \makecell{Signature\\Size} & \bfseries \makecell{Public Key\\Size} & \bfseries \makecell{Suitable\\Pre-Hashing} \\
				&&&&&&\\
				\hline\\

				Falcon-512
				& \tbd
				& \hspace{3mm}\doubleicon[themewhite]{\montserratbold I}{\faSun[regular]}{themered!65!black}{0.6}
				& \hspace{3mm}\tripleicon{\montserratbold 8}{\faMicrochip}{themered}{0.6}{\faKey}
				\tripleicon{\montserratbold 4}{\faMicrochip}{themeyellow}{0.6}{\faPen}
				\tripleicon{\montserratbold 2}{\faMicrochip}{themegreen}{0.6}{\faQuestionCircle}
				& \hspace{3mm}\doubleicon{\montserratbold 0}{\faPen}{themegreen}{0.6}
				& \hspace{3mm}\doubleicon{\montserratbold 4}{\faKey}{themeyellow}{0.6}
				& \tbd\\

				Falcon-1024
				& \tbd
				& \hspace{3mm}\doubleicon{\montserratbold V}{\faSun[regular]}{themegreen}{0.6}
				& \hspace{3mm}\tripleicon[themewhite]{\montserratbold 9}{\faMicrochip}{themered!50!black}{0.6}{\faKey}
				\tripleicon{\montserratbold 4}{\faMicrochip}{themeyellow}{0.6}{\faPen}
				\tripleicon{\montserratbold 3}{\faMicrochip}{themeyellow}{0.6}{\faQuestionCircle}
				& \hspace{3mm}\doubleicon{\montserratbold 0}{\faPen}{themegreen}{0.6}
				& \hspace{3mm}\doubleicon{\montserratbold 5}{\faKey}{themeorange}{0.6}
				& \tbd\\
			\end{tabular}
		\end{minipage}\\[\baselineskip]
		\hrule
		\vspace{1\baselineskip}
		\begin{minipage}[t]{0.49\textwidth}
			\scriptsize \faThumbsUp\, {\bfseries / Pros / Use If:}
			\begin{itemize}[leftmargin=*]
				\setlength\itemsep{0em}
				\item Falcon has smaller signature sizes than ML-DSA:\\
				\doubleicon{\montserratbold 0}{\faPen}{themegreen}{0.6}\hspace{-2mm} vs. \doubleicon{\montserratbold 1}{\faPen}{themegreen}{0.6}\hspace{-2mm} resp. \doubleicon{\montserratbold 2}{\faPen}{themegreen}{0.6}\hspace{-2mm}
				\item Falcon has smaller public key sizes than ML-DSA:\\
				\doubleicon{\montserratbold 4}{\faKey}{themeyellow}{0.6}\hspace{-2mm} vs. \doubleicon{\montserratbold 5}{\faKey}{themeorange}{0.6}\hspace{-2mm} on Level I, \doubleicon{\montserratbold 5}{\faKey}{themeorange}{0.6}\hspace{-2mm} vs. \doubleicon{\montserratbold 6}{\faKey}{themeorange}{0.6}\hspace{-2mm} on Level V.
			\end{itemize}
		\end{minipage}
		\hfill
		\begin{minipage}[t]{0.49\textwidth}
			\scriptsize \faThumbsDown\, {\bfseries / Cons / Don't Use If:}
			\begin{itemize}[leftmargin=*]
			\setlength\itemsep{0em}
			\item You need a medium security category between \doubleicon[themewhite]{\montserratbold I}{\faSun[regular]}{themered!65!black}{0.6} \hspace{-4mm} and \doubleicon{\montserratbold V}{\faSun[regular]}{themegreen}{0.6}
			\item The algorithm is not yet standardized
			\item The algorithm requires expensive floating point arithmetic
			\item Key generation and signing are slower than for ML-DSA
			\end{itemize}

		\end{minipage}\\[\baselineskip]

		{\tiny  {\bfseries \scshape Note:} Cycle counts for key generation, signing and verification depend on the CPU used. Values may vary on different CPUs and thus only a rough indicator.}
		\vspace{0cm}
	\end{algorithmbox}

	% ============================================================================================================
	% SLH-DSA
	% ============================================================================================================
	\begin{algorithmbox}{SLH-DSA (Stateless Hash-Based Digital Signature Standard)}
		\begin{minipage}[t]{0.3\textwidth}
			\scriptsize
			\begin{center}
				\adjustbox{valign=t}{%
					\GreenAbsoluteSpeedometer(0,-10)[2.25cm][0.75]{Unusable/1-3,Poor/4-5,Good/6-8,Perfect/9-10}{Algorithm Overall Usability Score}{5.26}{10}{\tiny}{\tiny \montserratsemibold}\hspace{0.5cm}}
			\end{center}
			\begin{tabbing}
				\bfseries \scshape Previous Name\hspace{2cm}\=SPHINCS+\\
				\bfseries \scshape Specification\>\iconhref[\faLink]{https://csrc.nist.gov/pubs/fips/205/final}{FIPS 205}\\
				\bfseries \scshape Type\>Signature\\
				\bfseries \scshape Family\>Hash (stateless)\\
				\bfseries \scshape Standardization Status\>Standardized\\
				\bfseries \scshape Recommended By\>NIST, BSI, ANSSI\\
				\bfseries \scshape Hashing\>Pure, Pre-Hashing\\
				\bfseries \scshape Naming\>based on various characteristics\\
				\>(*s=small signatures, *f=fast)
			\end{tabbing}
			\begin{tabular}[t]{l c  c  c}
				\scshape\bfseries Algorithm Implementation &\textcolor{themegreydark}{\faKey}&\textcolor{themegreydark}{\faPen}&\textcolor{themegreydark}{\faQuestionCircle}\\
				&&&\\
				\hline\\


				\scshape Complexity
				&\hspace{3mm}\tripleicon[themewhite]{\montserratbold ?}{\faCode}{themeaccentsecondary}{0.6}{\bfseries C}
				&\hspace{3mm}\tripleicon[themewhite]{\montserratbold ?}{\faCode}{themeaccentsecondary}{0.6}{\bfseries C}
				&\hspace{3mm}\tripleicon[themewhite]{\montserratbold ?}{\faCode}{themeaccentsecondary}{0.6}{\bfseries C}\\[2mm]
				\scshape Size
				&\hspace{3mm}\tripleicon[themewhite]{\montserratbold ?}{\faCode}{themeaccentsecondary}{0.6}{\bfseries S}
				&\hspace{3mm}\tripleicon[themewhite]{\montserratbold ?}{\faCode}{themeaccentsecondary}{0.6}{\bfseries S}
				&\hspace{3mm}\tripleicon[themewhite]{\montserratbold ?}{\faCode}{themeaccentsecondary}{0.6}{\bfseries S}\\
			\end{tabular}\\[1.5\baselineskip]
		\end{minipage}
		\hfill
		\begin{minipage}[t]{0.68\textwidth}
			\scshape \scriptsize
			\begin{tabular}[t]{c c c c c c c}
				\bfseries \makecell{Version\\{}} &  \bfseries \makecell{OID\\{}} &\bfseries \makecell{Security\\Category} & \bfseries \makecell{Performance\\{\faKey\,\quad\quad\faPen\,\quad\quad\faQuestionCircle}} &  \bfseries \makecell{Signature\\Size} & \bfseries \makecell{Public Key\\Size} & \bfseries \makecell{Suitable\\Pre-Hashing} \\
				&&&&&&\\
				\hline\\

				SLH-DSA-SHA2-128s
				& 2.16.840.1.101.3.4.3.20
				& \hspace{3mm}\doubleicon[themewhite]{\montserratbold I}{\faSun[regular]}{themered!65!black}{0.6}
				& \hspace{3mm}\tripleicon{\montserratbold 8}{\faMicrochip}{themered}{0.6}{\faKey}
				\tripleicon[themewhite]{\montserratbold 9}{\faMicrochip}{themered!50!black}{0.6}{\faPen}
				\tripleicon{\montserratbold 5}{\faMicrochip}{themeorange}{0.6}{\faQuestionCircle}
				& \hspace{3mm}\doubleicon{\montserratbold 2}{\faPen}{themegreen}{0.6}
				& \hspace{3mm}\doubleicon{\montserratbold 0}{\faKey}{themegreen}{0.6}
				& SHA-256, SHA3-256\\

				SLH-DSA-SHA2-128f
				& 2.16.840.1.101.3.4.3.21
				& \hspace{3mm}\doubleicon[themewhite]{\montserratbold I}{\faSun[regular]}{themered!65!black}{0.6}
				& \hspace{3mm}\tripleicon{\montserratbold 5}{\faMicrochip}{themeorange}{0.6}{\faKey}
				\tripleicon{\montserratbold 7}{\faMicrochip}{themered}{0.6}{\faPen}
				\tripleicon{\montserratbold 5}{\faMicrochip}{themeorange}{0.6}{\faQuestionCircle}
				& \hspace{3mm}\doubleicon{\montserratbold 4}{\faPen}{themeyellow}{0.6}
				& \hspace{3mm}\doubleicon{\montserratbold 0}{\faKey}{themegreen}{0.6}
				& SHA-256, SHA3-256\\


				SLH-DSA-SHA2-192s
				& 2.16.840.1.101.3.4.3.22
				& \hspace{3mm}\doubleicon{\montserratbold III}{\faSun[regular]}{themeorange}{0.6}
				& \hspace{3mm}\tripleicon{\montserratbold 8}{\faMicrochip}{themered}{0.6}{\faKey}
				\tripleicon[themewhite]{\montserratbold 9}{\faMicrochip}{themered!50!black}{0.6}{\faPen}
				\tripleicon{\montserratbold 5}{\faMicrochip}{themeorange}{0.6}{\faQuestionCircle}
				& \hspace{3mm}\doubleicon{\montserratbold 4}{\faPen}{themeyellow}{0.6}
				& \hspace{3mm}\doubleicon{\montserratbold 0}{\faKey}{themegreen}{0.6}
				& SHA-384, SHA3-384\\

				SLH-DSA-SHA2-192f
				& 2.16.840.1.101.3.4.3.23
				& \hspace{3mm}\doubleicon{\montserratbold III}{\faSun[regular]}{themeorange}{0.6}
				& \hspace{3mm}\tripleicon{\montserratbold 5}{\faMicrochip}{themeorange}{0.6}{\faKey}
				\tripleicon{\montserratbold 7}{\faMicrochip}{themered}{0.6}{\faPen}
				\tripleicon{\montserratbold 6}{\faMicrochip}{themeorange}{0.6}{\faQuestionCircle}
				& \hspace{3mm}\doubleicon{\montserratbold 5}{\faPen}{themeorange}{0.6}
				& \hspace{3mm}\doubleicon{\montserratbold 1}{\faKey}{themegreen}{0.6}
				& SHA-384, SHA3-384\\


				SLH-DSA-SHA2-256s
				& 2.16.840.1.101.3.4.3.24
				& \hspace{3mm}\doubleicon{\montserratbold V}{\faSun[regular]}{themegreen}{0.6}
				& \hspace{3mm}\tripleicon{\montserratbold 7}{\faMicrochip}{themered}{0.6}{\faKey}
				\tripleicon[themewhite]{\montserratbold 9}{\faMicrochip}{themered!50!black}{0.6}{\faPen}
				\tripleicon{\montserratbold 5}{\faMicrochip}{themeorange}{0.6}{\faQuestionCircle}
				& \hspace{3mm}\doubleicon{\montserratbold 4}{\faPen}{themeyellow}{0.6}
				& \hspace{3mm}\doubleicon{\montserratbold 0}{\faKey}{themegreen}{0.6}
				& SHA-512, SHA3-512\\

				SLH-DSA-SHA2-256f
				& 2.16.840.1.101.3.4.3.25
				& \hspace{3mm}\doubleicon{\montserratbold V}{\faSun[regular]}{themegreen}{0.6}
				& \hspace{3mm}\tripleicon{\montserratbold 6}{\faMicrochip}{themeorange}{0.6}{\faKey}
				\tripleicon{\montserratbold 8}{\faMicrochip}{themered}{0.6}{\faPen}
				\tripleicon{\montserratbold 6}{\faMicrochip}{themeorange}{0.6}{\faQuestionCircle}
				& \hspace{3mm}\doubleicon{\montserratbold 5}{\faPen}{themeorange}{0.6}
				& \hspace{3mm}\doubleicon{\montserratbold 1}{\faKey}{themegreen}{0.6}
				& SHA-512, SHA3-512\\








				SLH-DSA-SHAKE-128s
				& 2.16.840.1.101.3.4.3.26
				& \hspace{3mm}\doubleicon[themewhite]{\montserratbold I}{\faSun[regular]}{themered!65!black}{0.6}
				& \hspace{3mm}\tripleicon{\montserratbold 8}{\faMicrochip}{themered}{0.6}{\faKey}
				\tripleicon[themewhite]{\montserratbold 9}{\faMicrochip}{themered!50!black}{0.6}{\faPen}
				\tripleicon{\montserratbold 5}{\faMicrochip}{themeorange}{0.6}{\faQuestionCircle}
				& \hspace{3mm}\doubleicon{\montserratbold 2}{\faPen}{themegreen}{0.6}
				& \hspace{3mm}\doubleicon{\montserratbold 0}{\faKey}{themegreen}{0.6}
				& SHA-256, SHA3-256\\

				SLH-DSA-SHAKE-128f
				& 2.16.840.1.101.3.4.3.27
				& \hspace{3mm}\doubleicon[themewhite]{\montserratbold I}{\faSun[regular]}{themered!65!black}{0.6}
				& \hspace{3mm}\tripleicon{\montserratbold 5}{\faMicrochip}{themeorange}{0.6}{\faKey}
				\tripleicon{\montserratbold 7}{\faMicrochip}{themered}{0.6}{\faPen}
				\tripleicon{\montserratbold 5}{\faMicrochip}{themeorange}{0.6}{\faQuestionCircle}
				& \hspace{3mm}\doubleicon{\montserratbold 4}{\faPen}{themeyellow}{0.6}
				& \hspace{3mm}\doubleicon{\montserratbold 0}{\faKey}{themegreen}{0.6}
				& SHA-256, SHA3-256\\


				SLH-DSA-SHAKE-192s
				& 2.16.840.1.101.3.4.3.28
				& \hspace{3mm}\doubleicon{\montserratbold III}{\faSun[regular]}{themeorange}{0.6}
				& \hspace{3mm}\tripleicon{\montserratbold 8}{\faMicrochip}{themered}{0.6}{\faKey}
				\tripleicon[themewhite]{\montserratbold 9}{\faMicrochip}{themered!50!black}{0.6}{\faPen}
				\tripleicon{\montserratbold 5}{\faMicrochip}{themeorange}{0.6}{\faQuestionCircle}
				& \hspace{3mm}\doubleicon{\montserratbold 4}{\faPen}{themeyellow}{0.6}
				& \hspace{3mm}\doubleicon{\montserratbold 0}{\faKey}{themegreen}{0.6}
				& SHA-384, SHA3-384\\

				SLH-DSA-SHAKE-192f
				& 2.16.840.1.101.3.4.3.29
				& \hspace{3mm}\doubleicon{\montserratbold III}{\faSun[regular]}{themeorange}{0.6}
				& \hspace{3mm}\tripleicon{\montserratbold 5}{\faMicrochip}{themeorange}{0.6}{\faKey}
				\tripleicon{\montserratbold 7}{\faMicrochip}{themered}{0.6}{\faPen}
				\tripleicon{\montserratbold 6}{\faMicrochip}{themeorange}{0.6}{\faQuestionCircle}
				& \hspace{3mm}\doubleicon{\montserratbold 5}{\faPen}{themeorange}{0.6}
				& \hspace{3mm}\doubleicon{\montserratbold 1}{\faKey}{themegreen}{0.6}
				& SHA-384, SHA3-384\\


				SLH-DSA-SHAKE-256s
				& 2.16.840.1.101.3.4.3.30
				& \hspace{3mm}\doubleicon{\montserratbold V}{\faSun[regular]}{themegreen}{0.6}
				& \hspace{3mm}\tripleicon{\montserratbold 7}{\faMicrochip}{themered}{0.6}{\faKey}
				\tripleicon[themewhite]{\montserratbold 9}{\faMicrochip}{themered!50!black}{0.6}{\faPen}
				\tripleicon{\montserratbold 5}{\faMicrochip}{themeorange}{0.6}{\faQuestionCircle}
				& \hspace{3mm}\doubleicon{\montserratbold 4}{\faPen}{themeyellow}{0.6}
				& \hspace{3mm}\doubleicon{\montserratbold 0}{\faKey}{themegreen}{0.6}
				& SHA-512, SHA3-512\\

				SLH-DSA-SHAKE-256f
				& 2.16.840.1.101.3.4.3.31
				& \hspace{3mm}\doubleicon{\montserratbold V}{\faSun[regular]}{themegreen}{0.6}
				& \hspace{3mm}\tripleicon{\montserratbold 6}{\faMicrochip}{themeorange}{0.6}{\faKey}
				\tripleicon{\montserratbold 8}{\faMicrochip}{themered}{0.6}{\faPen}
				\tripleicon{\montserratbold 6}{\faMicrochip}{themeorange}{0.6}{\faQuestionCircle}
				& \hspace{3mm}\doubleicon{\montserratbold 5}{\faPen}{themeorange}{0.6}
				& \hspace{3mm}\doubleicon{\montserratbold 1}{\faKey}{themegreen}{0.6}
				& SHA-512, SHA3-512\\
			\end{tabular}
		\end{minipage}\\[\baselineskip]
		\hrule
		\vspace{1\baselineskip}
		\begin{minipage}[t]{0.49\textwidth}
			\scriptsize\faThumbsUp\, {\bfseries / Pros / Use If:}
			\begin{itemize}[leftmargin=*]
				\setlength\itemsep{0em}
				\item Alternative to ML-DSA and Falcon that is not based on lattices
				\item Very small public keys
			\end{itemize}
		\end{minipage}
		\hfill
		\begin{minipage}[t]{0.49\textwidth}
			\scriptsize \faThumbsDown\, {\bfseries / Cons / Don't Use If:}
			\begin{itemize}[leftmargin=*]
				\setlength\itemsep{0em}
				\item Poor key generation and signing performance compared to other algorithms
				\item High complexity of the algorithm and the implementation
				\item Possible interoperability issues due to the many variants that may not all be supported everywhere
			\end{itemize}
		\end{minipage}\\[\baselineskip]

		{\tiny  {\bfseries \scshape Note:} Cycle counts for key generation, signing and verification depend on the CPU used. Values may vary on different CPUs and thus only a rough indicator.}
	\end{algorithmbox}



	% ============================================================================================================
	% XMSS/LMS
	% ============================================================================================================
	\begin{algorithmbox}{XMSS / XMSS-MT (eXtended Merkle Signature Scheme / eXtended Merkle Signature Scheme Multi Tree)}
		\begin{minipage}[t]{0.35\textwidth}
			\scriptsize
			\begin{center}
				\adjustbox{valign=t}{%
					\GreenAbsoluteSpeedometer(0,-10)[2.25cm][0.75]{Unusable/1-3,Poor/4-5,Good/6-8,Perfect/9-10}{Algorithm Overall Usability Score}{0}{10}{\tiny}{\tiny \montserratsemibold}\hspace{1.5cm}}
			\end{center}
			\begin{tabbing}
				\bfseries \scshape Previous Name\hspace{2.5cm}\=XMSS/XMSSMT\\
				\bfseries \scshape Specification\>\iconhref[\faLink]{https://nvlpubs.nist.gov/nistpubs/SpecialPublications/NIST.SP.800-208.pdf}{SP 800-208}, \iconhref[\faLink]{https://datatracker.ietf.org/doc/html/rfc8391}{RFC 8391}\\


				\bfseries \scshape Type\>Signature\\
				\bfseries \scshape Family\>Merkle Trees (stateful hash trees)\\
				\bfseries \scshape Standardization Status\>Standardized\\
				\bfseries \scshape Recommended By\>NIST, BSI, ANSSI\\
				\bfseries \scshape Hashing\>\tbd\\
				\bfseries \scshape Naming\>XMSS-[Hashfamily]\_[h]\_[n]\\
				\>XMSSMT-[Hashfamily]\_[h]/[d]\_[n]\\
				\> where h is the tree height,\\
				\> d is the number of layers, and\\
				\> n is the message length in bits
			\end{tabbing}
			\begin{tabular}[t]{l c  c  c}
				\scshape\bfseries Algorithm Implementation &\textcolor{themegreydark}{\faKey}&\textcolor{themegreydark}{\faPen}&\textcolor{themegreydark}{\faQuestionCircle}\\
				&&&\\
				\hline\\


				\scshape Complexity
				&\hspace{3mm}\tripleicon[themewhite]{\montserratbold ?}{\faCode}{themeaccentsecondary}{0.6}{\bfseries C}
				&\hspace{3mm}\tripleicon[themewhite]{\montserratbold ?}{\faCode}{themeaccentsecondary}{0.6}{\bfseries C}
				&\hspace{3mm}\tripleicon[themewhite]{\montserratbold ?}{\faCode}{themeaccentsecondary}{0.6}{\bfseries C}\\[2mm]
				\scshape Size
				&\hspace{3mm}\tripleicon[themewhite]{\montserratbold ?}{\faCode}{themeaccentsecondary}{0.6}{\bfseries S}
				&\hspace{3mm}\tripleicon[themewhite]{\montserratbold ?}{\faCode}{themeaccentsecondary}{0.6}{\bfseries S}
				&\hspace{3mm}\tripleicon[themewhite]{\montserratbold ?}{\faCode}{themeaccentsecondary}{0.6}{\bfseries S}
			\end{tabular}\\[1.5\baselineskip]
		\end{minipage}
		\hfill
		\begin{minipage}[t]{0.64\textwidth}
			\scshape \scriptsize
			\begin{tabular}[t]{c c  c  c  c  c  c}
				\bfseries \makecell{Version\\{}} &  \bfseries \makecell{Numeric\\Identifier} &\bfseries \makecell{Security\\Category} & \bfseries \makecell{Performance\\{\faKey\,\quad\quad\faPen\,\quad\quad\faQuestionCircle}} &  \bfseries \makecell{Signature\\Size} & \bfseries \makecell{Maxiumum \\Signatures} & \bfseries \makecell{Number of\\layers} \\
				&&&&&&\\
				\hline\\


				XMSS-SHA2\_10\_256
				& 0x00000001
				& \hspace{3mm}\doubleicon{\montserratbold V}{\faSun[regular]}{themegreen}{0.6}
				& \hspace{3mm}\tripleicon[themewhite]{\montserratbold ?}{\faMicrochip}{themeaccentsecondary}{0.6}{\faKey}
				\tripleicon[themewhite]{\montserratbold ?}{\faMicrochip}{themeaccentsecondary}{0.6}{\faPen}
				\tripleicon[themewhite]{\montserratbold ?}{\faMicrochip}{themeaccentsecondary}{0.6}{\faQuestionCircle}
				& \hspace{3mm}\doubleicon[themewhite]{\montserratbold ?}{\faMicrochip}{themeaccentsecondary}{0.6}
				& 2\textsuperscript{10}
				& 1\\

				XMSS-SHA2\_16\_256
				& 0x00000002
				& \hspace{3mm}\doubleicon{\montserratbold V}{\faSun[regular]}{themegreen}{0.6}
				& \hspace{3mm}\tripleicon[themewhite]{\montserratbold ?}{\faMicrochip}{themeaccentsecondary}{0.6}{\faKey}
				\tripleicon[themewhite]{\montserratbold ?}{\faMicrochip}{themeaccentsecondary}{0.6}{\faPen}
				\tripleicon[themewhite]{\montserratbold ?}{\faMicrochip}{themeaccentsecondary}{0.6}{\faQuestionCircle}
				& \hspace{3mm}\doubleicon[themewhite]{\montserratbold ?}{\faMicrochip}{themeaccentsecondary}{0.6}
				& 2\textsuperscript{16}
				& 1\\

				XMSS-SHA2\_20\_256
				& 0x00000003
				& \hspace{3mm}\doubleicon{\montserratbold V}{\faSun[regular]}{themegreen}{0.6}
				& \hspace{3mm}\tripleicon[themewhite]{\montserratbold ?}{\faMicrochip}{themeaccentsecondary}{0.6}{\faKey}
				\tripleicon[themewhite]{\montserratbold ?}{\faMicrochip}{themeaccentsecondary}{0.6}{\faPen}
				\tripleicon[themewhite]{\montserratbold ?}{\faMicrochip}{themeaccentsecondary}{0.6}{\faQuestionCircle}
				& \hspace{3mm}\doubleicon[themewhite]{\montserratbold ?}{\faMicrochip}{themeaccentsecondary}{0.6}
				& 2\textsuperscript{20}
				& 1\\[\baselineskip]
				&&&&&&\\

				\hline\\
				XMSSMT-SHA2\_20/2\_256
				& 0x00000001
				& \hspace{3mm}\doubleicon{\montserratbold V}{\faSun[regular]}{themegreen}{0.6}
				& \hspace{3mm}\tripleicon[themewhite]{\montserratbold ?}{\faMicrochip}{themeaccentsecondary}{0.6}{\faKey}
				\tripleicon[themewhite]{\montserratbold ?}{\faMicrochip}{themeaccentsecondary}{0.6}{\faPen}
				\tripleicon[themewhite]{\montserratbold ?}{\faMicrochip}{themeaccentsecondary}{0.6}{\faQuestionCircle}
				& \hspace{3mm}\doubleicon[themewhite]{\montserratbold ?}{\faMicrochip}{themeaccentsecondary}{0.6}
				& 2\textsuperscript{20}
				& 2\\

				XMSSMT-SHA2\_20/4\_256
				& 0x00000002
				& \hspace{3mm}\doubleicon{\montserratbold V}{\faSun[regular]}{themegreen}{0.6}
				& \hspace{3mm}\tripleicon[themewhite]{\montserratbold ?}{\faMicrochip}{themeaccentsecondary}{0.6}{\faKey}
				\tripleicon[themewhite]{\montserratbold ?}{\faMicrochip}{themeaccentsecondary}{0.6}{\faPen}
				\tripleicon[themewhite]{\montserratbold ?}{\faMicrochip}{themeaccentsecondary}{0.6}{\faQuestionCircle}
				& \hspace{3mm}\doubleicon[themewhite]{\montserratbold ?}{\faMicrochip}{themeaccentsecondary}{0.6}
				& 2\textsuperscript{20}
				& 4\\

				XMSSMT-SHA2\_40/2\_256
				& 0x00000003
				& \hspace{3mm}\doubleicon{\montserratbold V}{\faSun[regular]}{themegreen}{0.6}
				& \hspace{3mm}\tripleicon[themewhite]{\montserratbold ?}{\faMicrochip}{themeaccentsecondary}{0.6}{\faKey}
				\tripleicon[themewhite]{\montserratbold ?}{\faMicrochip}{themeaccentsecondary}{0.6}{\faPen}
				\tripleicon[themewhite]{\montserratbold ?}{\faMicrochip}{themeaccentsecondary}{0.6}{\faQuestionCircle}
				& \hspace{3mm}\doubleicon[themewhite]{\montserratbold ?}{\faMicrochip}{themeaccentsecondary}{0.6}
				& 2\textsuperscript{40}
				& 2\\

				XMSSMT-SHA2\_40/4\_256
				& 0x00000004
				& \hspace{3mm}\doubleicon{\montserratbold V}{\faSun[regular]}{themegreen}{0.6}
				& \hspace{3mm}\tripleicon[themewhite]{\montserratbold ?}{\faMicrochip}{themeaccentsecondary}{0.6}{\faKey}
				\tripleicon[themewhite]{\montserratbold ?}{\faMicrochip}{themeaccentsecondary}{0.6}{\faPen}
				\tripleicon[themewhite]{\montserratbold ?}{\faMicrochip}{themeaccentsecondary}{0.6}{\faQuestionCircle}
				& \hspace{3mm}\doubleicon[themewhite]{\montserratbold ?}{\faMicrochip}{themeaccentsecondary}{0.6}
				& 2\textsuperscript{40}
				& 4\\

				XMSSMT-SHA2\_40/8\_256
				& 0x00000005
				& \hspace{3mm}\doubleicon{\montserratbold V}{\faSun[regular]}{themegreen}{0.6}
				& \hspace{3mm}\tripleicon[themewhite]{\montserratbold ?}{\faMicrochip}{themeaccentsecondary}{0.6}{\faKey}
				\tripleicon[themewhite]{\montserratbold ?}{\faMicrochip}{themeaccentsecondary}{0.6}{\faPen}
				\tripleicon[themewhite]{\montserratbold ?}{\faMicrochip}{themeaccentsecondary}{0.6}{\faQuestionCircle}
				& \hspace{3mm}\doubleicon[themewhite]{\montserratbold ?}{\faMicrochip}{themeaccentsecondary}{0.6}
				& 2\textsuperscript{40}
				& 8\\

				XMSSMT-SHA2\_60/3\_256
				& 0x00000006
				& \hspace{3mm}\doubleicon{\montserratbold V}{\faSun[regular]}{themegreen}{0.6}
				& \hspace{3mm}\tripleicon[themewhite]{\montserratbold ?}{\faMicrochip}{themeaccentsecondary}{0.6}{\faKey}
				\tripleicon[themewhite]{\montserratbold ?}{\faMicrochip}{themeaccentsecondary}{0.6}{\faPen}
				\tripleicon[themewhite]{\montserratbold ?}{\faMicrochip}{themeaccentsecondary}{0.6}{\faQuestionCircle}
				& \hspace{3mm}\doubleicon[themewhite]{\montserratbold ?}{\faMicrochip}{themeaccentsecondary}{0.6}
				& 2\textsuperscript{60}
				& 3\\

				XMSSMT-SHA2\_60/6\_256
				& 0x00000007
				& \hspace{3mm}\doubleicon{\montserratbold V}{\faSun[regular]}{themegreen}{0.6}
				& \hspace{3mm}\tripleicon[themewhite]{\montserratbold ?}{\faMicrochip}{themeaccentsecondary}{0.6}{\faKey}
				\tripleicon[themewhite]{\montserratbold ?}{\faMicrochip}{themeaccentsecondary}{0.6}{\faPen}
				\tripleicon[themewhite]{\montserratbold ?}{\faMicrochip}{themeaccentsecondary}{0.6}{\faQuestionCircle}
				& \hspace{3mm}\doubleicon[themewhite]{\montserratbold ?}{\faMicrochip}{themeaccentsecondary}{0.6}
				& 2\textsuperscript{60}
				& 6\\

				XMSSMT-SHA2\_60/12\_256
				& 0x00000008
				& \hspace{3mm}\doubleicon{\montserratbold V}{\faSun[regular]}{themegreen}{0.6}
				& \hspace{3mm}\tripleicon[themewhite]{\montserratbold ?}{\faMicrochip}{themeaccentsecondary}{0.6}{\faKey}
				\tripleicon[themewhite]{\montserratbold ?}{\faMicrochip}{themeaccentsecondary}{0.6}{\faPen}
				\tripleicon[themewhite]{\montserratbold ?}{\faMicrochip}{themeaccentsecondary}{0.6}{\faQuestionCircle}
				& \hspace{3mm}\doubleicon[themewhite]{\montserratbold ?}{\faMicrochip}{themeaccentsecondary}{0.6}
				& 2\textsuperscript{60}
				& 12
			\end{tabular}\\[2.5\baselineskip]

			{\bfseries Note:}\\
			\normalfont\iconhref[\faLink]{https://nvlpubs.nist.gov/nistpubs/SpecialPublications/NIST.SP.800-208.pdf}{SP 800-208} defines further parameter sets not listed in \iconhref[\faLink]{https://datatracker.ietf.org/doc/html/rfc8391}{RFC 8391} using other hash functions (SHA256/192, SHAKE256/256, SHAKE256/192). Furthermore, \iconhref[\faLink]{https://datatracker.ietf.org/doc/html/rfc8391}{RFC 8391} lists optional parameter sets that are not approved in\\ \iconhref[\faLink]{https://nvlpubs.nist.gov/nistpubs/SpecialPublications/NIST.SP.800-208.pdf}{SP 800-208}. All of those variants are omitted here as they are not likely to be widely used, in particular not after ML-DSA and SLH-DSA have been standardized in the meantime.
		\end{minipage}\\
		\hrule
		\vspace{1\baselineskip}
		\begin{minipage}[t]{0.49\textwidth}
			\scriptsize\faThumbsUp\, {\bfseries / Pros / Use If:}
			\begin{itemize}[leftmargin=*]
				\setlength\itemsep{0em}
				\item You can predict the maximum number of signatures that are going to be required
				\item Firmware signing use cases
				\item You want a signature scheme where the security only relies on the security of the hash function used without assuming the hardness of another mathematical problem.
				\item Cf. \iconhref[\faLink]{https://nvlpubs.nist.gov/nistpubs/SpecialPublications/NIST.SP.800-208.pdf}{SP 800-208, Section 1.1} for additional explanations
			\end{itemize}
		\end{minipage}
		\hfill
		\begin{minipage}[t]{0.49\textwidth}
			\scriptsize \faThumbsDown\, {\bfseries / Cons / Don't Use If:}
			\begin{itemize}[leftmargin=*]
				\setlength\itemsep{0em}
				\item You require an algorithm for general use
				\item You cannot predict the maximum number of signatures that are going to be required, or the number of required signatures exceeds the maximum number of signatures enabled through the approved parameter sets
				\item Your application does not allow for the careful state management and tracking of signatures performed that is required with this algorithm
			\end{itemize}
		\end{minipage}\\[\baselineskip]

		{\tiny  {\bfseries \scshape Note:} Cycle counts for key generation, signing and verification depend on the CPU used. Values may vary on different CPUs and thus only a rough indicator.}
	\end{algorithmbox}
	\begin{algorithmbox}{LMS (Leighton-Micali Signature)}
		\textcolor{themeblue}{TBD}
	\end{algorithmbox}


	% ============================================================================================================
	% ML-KEM
	% ============================================================================================================
	\newpage
	{\large \scshape \faLock\, Encryption Algorithm Overview \& ID Cards}\\

	\begin{algorithmbox}{ML-KEM (Module-Lattice-Based Key-Encapsulation Mechanism Standard)}
	\begin{minipage}[t]{0.38\textwidth}
			\scriptsize
			\begin{center}
				\adjustbox{valign=t}{%
					\GreenAbsoluteSpeedometer(0,-10)[2.25cm][0.75]{Unusable/1-3,Poor/4-5,Good/6-8,Perfect/9-10}{Algorithm Overall Usability Score}{8.28}{10}{\tiny}{\tiny \montserratsemibold}\hspace{1.5cm}}
			\end{center}
			\begin{tabbing}
				\bfseries \scshape Previous Name\hspace{2.5cm}\=CRYSTALS-KYBER\\
				\bfseries \scshape Specification\>\iconhref[\faLink]{https://csrc.nist.gov/pubs/fips/203/final}{FIPS 203}\\
				\bfseries \scshape Type\>Encryption/KEM\\
				\bfseries \scshape Family\>Lattice\\
				\bfseries \scshape Standardization Status\>Standardized\\
				\bfseries \scshape Recommended By\>NIST, BSI, ANSSI\\
				\bfseries \scshape Naming\>\tbd
			\end{tabbing}
			\begin{tabular}[t]{l c  c  c}
				\scshape\bfseries Algorithm Implementation &\textcolor{themegreydark}{\faKey}&\textcolor{themegreydark}{\faPen}&\textcolor{themegreydark}{\faQuestionCircle}\\
				&&&\\
				\hline\\


				\scshape Complexity
				&\hspace{3mm}\tripleicon[themewhite]{\montserratbold ?}{\faCode}{themeaccentsecondary}{0.6}{\bfseries C}
				&\hspace{3mm}\tripleicon[themewhite]{\montserratbold ?}{\faCode}{themeaccentsecondary}{0.6}{\bfseries C}
				&\hspace{3mm}\tripleicon[themewhite]{\montserratbold ?}{\faCode}{themeaccentsecondary}{0.6}{\bfseries C}\\[2mm]
				\scshape Size
				&\hspace{3mm}\tripleicon[themewhite]{\montserratbold ?}{\faCode}{themeaccentsecondary}{0.6}{\bfseries S}
				&\hspace{3mm}\tripleicon[themewhite]{\montserratbold ?}{\faCode}{themeaccentsecondary}{0.6}{\bfseries S}
				&\hspace{3mm}\tripleicon[themewhite]{\montserratbold ?}{\faCode}{themeaccentsecondary}{0.6}{\bfseries S}\\
			\end{tabular}\\[1.5\baselineskip]
		\end{minipage}
		\hfill
		\begin{minipage}[t]{0.6\textwidth}
			\scshape \scriptsize
			\begin{tabular}[t]{c c  c  c  c  c}
				\bfseries \makecell{Version\\{}} &  \bfseries \makecell{OID\\{}} &\bfseries \makecell{Security\\Category} & \bfseries \makecell{Performance\\{\faKey\,\quad\quad\faLock\,\quad\quad\faUnlock}} &  \bfseries \makecell{Ciphertext\\Size} & \bfseries \makecell{Public Key\\Size}\\
				&&&&&\\
				\hline\\


				ML-KEM-512
				& 2.16.840.1.101.3.4.4.7
				& \hspace{3mm}\doubleicon[themewhite]{\montserratbold I}{\faSun[regular]}{themered!65!black}{0.6}
				& \hspace{3mm}\tripleicon{\montserratbold 2}{\faMicrochip}{themegreen}{0.6}{\faKey}
				\tripleicon{\montserratbold 2}{\faMicrochip}{themegreen}{0.6}{\faLock}
				\tripleicon{\montserratbold 2}{\faMicrochip}{themegreen}{0.6}{\faUnlock}
				& \hspace{3mm}\doubleicon{\montserratbold 0}{\faLock}{themegreen}{0.6}
				& \hspace{3mm}\doubleicon{\montserratbold 0}{\faKey}{themegreen}{0.6}\\

				ML-KEM-768
				& 2.16.840.1.101.3.4.4.2
				& \hspace{3mm}\doubleicon{\montserratbold III}{\faSun[regular]}{themeyellow}{0.6}
				& \hspace{3mm}\tripleicon{\montserratbold 2}{\faMicrochip}{themegreen}{0.6}{\faKey}
				\tripleicon{\montserratbold 2}{\faMicrochip}{themegreen}{0.6}{\faLock}
				\tripleicon{\montserratbold 3}{\faMicrochip}{themeyellow}{0.6}{\faUnlock}
				& \hspace{3mm}\doubleicon{\montserratbold 0}{\faLock}{themegreen}{0.6}
				& \hspace{3mm}\doubleicon{\montserratbold 0}{\faKey}{themegreen}{0.6}\\

				ML-KEM-1024
				& 2.16.840.1.101.3.4.4.3
				& \hspace{3mm}\doubleicon{\montserratbold V}{\faSun[regular]}{themegreen}{0.6}
				& \hspace{3mm}\tripleicon{\montserratbold 3}{\faMicrochip}{themeyellow}{0.6}{\faKey}
				\tripleicon{\montserratbold 3}{\faMicrochip}{themeyellow}{0.6}{\faLock}
				\tripleicon{\montserratbold 3}{\faMicrochip}{themeyellow}{0.6}{\faUnlock}
				& \hspace{3mm}\doubleicon{\montserratbold 0}{\faLock}{themegreen}{0.6}
				& \hspace{3mm}\doubleicon{\montserratbold 0}{\faKey}{themegreen}{0.6}\\
			\end{tabular}
		\end{minipage}\\[\baselineskip]
		\hrule
		\vspace{1\baselineskip}
		\begin{minipage}[t]{0.49\textwidth}
			\scriptsize\faThumbsUp\, {\bfseries / Pros / Use If:}
			\begin{itemize}[leftmargin=*]
				\setlength\itemsep{0em}
				\item Currently the only post-quantum encryption /key encapsulation algorithm standardized by NIST
				\item Need a general-purpose encryption / key-encapsulation algorithm with decent specs in all categories
			\end{itemize}
		\end{minipage}
		\hfill
		\begin{minipage}[t]{0.49\textwidth}
			\scriptsize \faThumbsDown\, {\bfseries / Cons / Don't Use If:}
			\begin{itemize}[leftmargin=*]
				\setlength\itemsep{0em}
				\item You don't want a lattice-based algorithm
			\end{itemize}
		\end{minipage}\\[\baselineskip]

		{\tiny  {\bfseries \scshape Note:} Cycle counts for key generation, encryption and decryption depend on the CPU used. Values may vary on different CPUs and thus only a rough indicator.}
	\end{algorithmbox}


	\begin{algorithmbox}{Classic McEliece}
		\textcolor{themeblue}{TBD}


		{\tiny  {\bfseries \scshape Note:} Cycle counts for key generation, encryption and decryption depend on the CPU used. Values may vary on different CPUs and thus only a rough indicator.}
	\end{algorithmbox}

	\begin{algorithmbox}{BIKE}
		\textcolor{themeblue}{TBD}\\
		{\tiny  {\bfseries \scshape Note:} No data available for BIKE-L5 for cycle counts. Algorithm score is computed over BIKE-l1 and BIKE-L3 only.}\\
		{\tiny  {\bfseries \scshape Note:} Cycle counts for key generation, encryption and decryption depend on the CPU used. Values may vary on different CPUs and thus only a rough indicator.}

	\end{algorithmbox}

	\begin{algorithmbox}{HQC}
		\textcolor{themeblue}{TBD}\\
		{\tiny  {\bfseries \scshape Note:} Cycle counts for key generation, encryption and decryption depend on the CPU used. Values may vary on different CPUs and thus only a rough indicator.}
	\end{algorithmbox}

	\begin{algorithmbox}{FrodoKEM}
		\textcolor{themeblue}{TBD}\\
		{\tiny  {\bfseries \scshape Note:} Cycle counts for key generation, encryption and decryption depend on the CPU used. Values may vary on different CPUs and thus only a rough indicator.}
	\end{algorithmbox}


	\newpage
	\begin{algorithmbox}{Algorithm Scoring: Algorithm Overall Usability Score}
		\scriptsize
		We try to measure an algorithm's overall usability by calculating a single number between 0 (worst) and 10 (best), taking into account all performance and size metrics, the number of security categories provided and whether or not is it is suitable for general use.\\

		An algorithm's overall score is computed as
		\begin{align*}
			\textnormal{score\textsubscript{\itshape algorithm}} = \max\left\{0;\,\avg\{\textnormal{score}_\textnormal{\itshape v}\,|\, \textnormal{{\itshape v} is variant of {\itshape algorithm}}\} - \frac{1}{8}\cdot\left(5-\gamma_\textnormal{\itshape algorithm}\right) - \delta_\textnormal{\itshape algorithm}\right\}
		\end{align*}
		where \textnormal{score\textsubscript{\itshape variant}} is a score for an individual algorithm variant computed as
				\begin{align*}
				\textnormal{score\textsubscript{\itshape signature-variant}} = 10 - \avg\left({\scriptsize
				\tripleicon{\montserratbold n}{\faMicrochip}{themegreylight!50}{0.6}{\faKey} + %
				\tripleicon{\montserratbold n}{\faMicrochip}{themegreylight!50}{0.6}{\faPen} + %
				\tripleicon{\montserratbold n}{\faMicrochip}{themegreylight!50}{0.6}{\faQuestionCircle} + %
				\doubleicon{\montserratbold n}{\faPen}{themegreylight!50}{0.6}+ %
				\doubleicon{\montserratbold n}{\faKey}{themegreylight!50}{0.6}
				}\right)
			\end{align*}
		respectively
			\begin{align*}
				\textnormal{score\textsubscript{\itshape encryption-variant}} = 10 - \avg\left({\scriptsize
				\tripleicon{\montserratbold n}{\faMicrochip}{themegreylight!50}{0.6}{\faKey} + %
				\tripleicon{\montserratbold n}{\faMicrochip}{themegreylight!50}{0.6}{\faLock} + %
				\tripleicon{\montserratbold n}{\faMicrochip}{themegreylight!50}{0.6}{\faUnlock} + %
				\doubleicon{\montserratbold n}{\faLock}{themegreylight!50}{0.6}+ %
				\doubleicon{\montserratbold n}{\faKey}{themegreylight!50}{0.6}
				}\right)
			\end{align*}

		and where $1\leq\gamma_\textnormal{\itshape algorithm}\leq 5$ describes the number of different security categories offered by {\itshape algorithm}. Finally, $\delta_\textnormal{\itshape algorithm} = \begin{cases}
		0 & \textnormal{if {\itshape algorithm} is a general purpose algorithm} \\
		2 & \textnormal{else}
	\end{cases}$	takes into account if the algorithm is suitable for general use.\\

	\textcolor{themeaccentsecondary}{TBD: Improve formular by taking into account implementation complexity and size.}\\
   \end{algorithmbox}

   	\begin{algorithmbox}{Example: ML-DSA Overall Usability Score}
		\scriptsize
		We calculate
		\begin{align*}
			\textnormal{score\textsubscript{\itshape ML-DSA-44}} &= 10 - \avg\left({\scriptsize
			\tripleicon{\montserratbold 3}{\faMicrochip}{themeyellow}{0.6}{\faKey} + %
			\tripleicon{\montserratbold 3}{\faMicrochip}{themeyellow}{0.6}{\faPen} + %
			\tripleicon{\montserratbold 2}{\faMicrochip}{themegreen}{0.6}{\faQuestionCircle} + %
			\doubleicon{\montserratbold 1}{\faPen}{themegreen}{0.6}+ %
			\doubleicon{\montserratbold 5}{\faKey}{themeorange}{0.6}
			}\right) = 10 - 2.8 = 7.2
		\end{align*}
		Similarly, we obtain \textnormal{score\textsubscript{\itshape ML-DSA-65}} = 7 and \textnormal{score\textsubscript{\itshape ML-DSA-87}}=6.2. Furthermore, $\gamma_\textnormal{\itshape ML-DSA} = 3$ since ML-DSA offers the three security categories \doubleicon{\montserratbold II}{\faSun[regular]}{themered}{0.6}\hspace{-1mm}, \doubleicon{\montserratbold III}{\faSun[regular]}{themeorange}{0.6}\hspace{-1mm}, and \doubleicon{\montserratbold V}{\faSun[regular]}{themegreen}{0.6}\hspace{-1mm}, and $\delta_\textnormal{\itshape ML-DSA} = 0$ since ML-DSA is a general purpose signature algorithm. This results in an overall usability score of 6.55:\\[\baselineskip]

		\begin{minipage}[T]{0.25\textwidth}
			\GreenAbsoluteSpeedometer(0,0)[2.25cm][0.75]{Unusable/1-3,Poor/4-5,Good/6-8,Perfect/9-10}{ML-DSA Overall Usability Score}{6.55}{10}{\tiny}{\tiny \montserratsemibold}
		\end{minipage}
		\hfill
		\begin{minipage}[T]{0.75\textwidth}
		\vspace{-\baselineskip}
			\begin{align*}
				\textnormal{score\textsubscript{\itshape ML-DSA}} &= \max\left\{(0;\,\avg\{\textnormal{score}_\textnormal{\itshape ML-DSA-44},\textnormal{score}_\textnormal{\itshape ML-DSA-65},\textnormal{score}_\textnormal{\itshape ML-DSA-87}\} - \frac{1}{8}\cdot\left(5-\gamma_\textnormal{\itshape ML-DSA}\right) - \delta_\textnormal{\itshape ML-DSA}\right\}\\
				&=\max\left\{0;\,\avg\{7.2;\,7;\,6.2\} - \frac{1}{8}\cdot\left(5-3\right) - 0\right\}\\
				&=\max\left\{0;\,6.8 - 0.25 - 0\right\}\\
				&=\max\left\{0;\,6.55\right\}\\
				&=6.55
			\end{align*}
		\end{minipage}

	\end{algorithmbox}
\end{document}
