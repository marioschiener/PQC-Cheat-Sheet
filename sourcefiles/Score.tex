% !TeX root = ../PQC_Cheat_Sheet.tex

% Author: Mario Schiener, 2024
% LinkedIn: https://www.linkedin.com/in/mario-schiener/
% Github: https://github.com/marioschiener

% Note that formulas are made to look mostly like normal text on purpose as the usual look of mathematics wouldn't look good with the rest of the document.
% Unfortunately, this makes the source code quite messy and unreadable since we need to use \textnormal excessively in order to be able to still utilize math environemts like align. So, this is done on prupose, not out of a lack of knowledge regarding formulas in LaTeX. Some vspacing and hspacing is needed to force the boxes onto one page.

\vspace{-5mm}
\begin{algorithmbox}{Algorithm Overall Usability Score}
	\tiny
	We try to measure an algorithm's overall usability for a general use case without any special characteristics by calculating a single number between 0 (worst) and 10 (best) for the algorithm. This calculation is taking into account all parameter sets, performance metrics, public key size, signature/ciphertext size, the number of security categories provided, whether or not it is suitable for general use, and the complexity and size of its implementation.\\

	We define an algorithm's overall score as
	\begin{align*}
		\textnormal{\bfseries score\textsubscript{\itshape \bfseries algorithm}} = \textnormal{\bfseries max}\left\{\textnormal{\bfseries 0, }\textnormal{\bfseries avg}\left\{\textnormal{\bfseries score}_\textnormal{\itshape \bfseries parameterSet}\,|\, \textnormal{{\itshape\bfseries  parameterSet} \bfseries is a parameter set of {\itshape algorithm}}\right\} - \frac{\textnormal{\bfseries 1}}{\textnormal{\bfseries 8}}\cdot\left(\textnormal{\bfseries 5}-\textnormal{\bfseries categories}_\textnormal{\itshape\bfseries  algorithm}\right) - \frac{\textnormal{\bfseries 1}}{\textnormal{\bfseries 4}}\cdot\textnormal{\bfseries impl\textsubscript{\itshape\bfseries algorithm}}- \textnormal{\bfseries generality}_\textnormal{\itshape\bfseries  algorithm}\right\}
	\end{align*}
	where \textnormal{score\textsubscript{\itshape parameterSet}} is a score for an individual algorithm parameter set. Depending on the type of algorithm, \textnormal{score\textsubscript{\itshape parameterSet}} is defined as
	\begin{align*}
		\textnormal{score\textsubscript{\itshape signature-parameterSet}} = \textnormal{10} - \avg\left\{{
		\tripleicon{\montserratbold n}{\faMicrochip}{themegreylight!50}{0.6}{\faKey}\hspace{-1mm}\textnormal{, } %
		\tripleicon{\montserratbold n}{\faMicrochip}{themegreylight!50}{0.6}{\faPen}\hspace{-1mm}\textnormal{, } %
		\tripleicon{\montserratbold n}{\faMicrochip}{themegreylight!50}{0.6}{\faQuestionCircle}\hspace{-1mm}\textnormal{, } %
		\doubleicon{\montserratbold n}{\faPen}{themegreylight!50}{0.6}\hspace{-1mm}\textnormal{, } %
		\doubleicon{\montserratbold n}{\faKey}{themegreylight!50}{0.6}\hspace{-1mm}
		}\right\}
		\hspace{2cm}\textnormal{respectively}\hspace{2cm}
		\textnormal{score\textsubscript{\itshape encryption-parameterSet}} = \textnormal{10} - \avg\left\{{
		\tripleicon{\montserratbold n}{\faMicrochip}{themegreylight!50}{0.6}{\faKey}\hspace{-1mm}\textnormal{, } %
		\tripleicon{\montserratbold n}{\faMicrochip}{themegreylight!50}{0.6}{\faLock}\hspace{-1mm}\textnormal{, } %
		\tripleicon{\montserratbold n}{\faMicrochip}{themegreylight!50}{0.6}{\faUnlock}\hspace{-1mm}\textnormal{, } %
		\doubleicon{\montserratbold n}{\faLock}{themegreylight!50}{0.6}\hspace{-1mm}\textnormal{, } %
		\doubleicon{\montserratbold n}{\faKey}{themegreylight!50}{0.6}\hspace{-1mm}
		}\right\}\textnormal{.}
	\end{align*}
	Furthermore, $\textnormal{1}\leq\textnormal{categories}_\textnormal{\itshape algorithm}\leq\textnormal{5}$ denotes the number of different NIST security categories offered by {\itshape algorithm}. By assigning numeric values of 0 = \quadicon{\montserratbold L}{\faCode}{themegreen}{0.6}{\bfseries C}{\faAsterisk}\hspace{-2mm}= \quadicon{\montserratbold L}{\faCode}{themegreen}{0.6}{\bfseries S}{\faAsterisk}\hspace{-2mm}, 1 = \quadicon{\montserratbold M}{\faCode}{themeyellow}{0.6}{\bfseries C}{\faAsterisk}\hspace{-2mm}= \quadicon{\montserratbold M}{\faCode}{themeyellow}{0.6}{\bfseries S}{\faAsterisk}\hspace{-2mm} and 2 = \quadicon{\montserratbold H}{\faCode}{themered}{0.6}{\bfseries C}{\faAsterisk}\hspace{-2mm}= \quadicon{\montserratbold H}{\faCode}{themered}{0.6}{\bfseries S}{\faAsterisk}\hspace{-2mm}, we set
	\begin{align*}
		\textnormal{impl\textsubscript{\itshape signature-algorithm}} = \avg\left\{{
		\quadicon{\montserratbold ?}{\faCode}{themegreylight!50}{0.6}{\bfseries C}{\faKey}\hspace{-1mm}\textnormal{, }
		\quadicon{\montserratbold ?}{\faCode}{themegreylight!50}{0.6}{\bfseries C}{\faPen}\hspace{-1mm}\textnormal{, }
		\quadicon{\montserratbold ?}{\faCode}{themegreylight!50}{0.6}{\bfseries C}{\faQuestion}\hspace{-1mm}\textnormal{, }
		\quadicon{\montserratbold ?}{\faCode}{themegreylight!50}{0.6}{\bfseries S}{\faKey}\hspace{-1mm}\textnormal{, }
		\quadicon{\montserratbold ?}{\faCode}{themegreylight!50}{0.6}{\bfseries S}{\faPen}\hspace{-1mm}\textnormal{, }
		\quadicon{\montserratbold ?}{\faCode}{themegreylight!50}{0.6}{\bfseries S}{\faQuestion}\hspace{-1mm}
		}\right\}\hspace{2cm}\textnormal{respectively}\hspace{2cm}
		\textnormal{impl\textsubscript{\itshape encryption-algorithm}} = \avg\left\{{
		\quadicon{\montserratbold ?}{\faCode}{themegreylight!50}{0.6}{\bfseries C}{\faKey}\hspace{-1mm}\textnormal{, }
		\quadicon{\montserratbold ?}{\faCode}{themegreylight!50}{0.6}{\bfseries C}{\faLock}\hspace{-1mm}\textnormal{, }
		\quadicon{\montserratbold ?}{\faCode}{themegreylight!50}{0.6}{\bfseries C}{\faUnlock}\hspace{-1mm}\textnormal{, }
		\quadicon{\montserratbold ?}{\faCode}{themegreylight!50}{0.6}{\bfseries S}{\faKey}\hspace{-1mm}\textnormal{, }
		\quadicon{\montserratbold ?}{\faCode}{themegreylight!50}{0.6}{\bfseries S}{\faLock}\hspace{-1mm}\textnormal{, }
		\quadicon{\montserratbold ?}{\faCode}{themegreylight!50}{0.6}{\bfseries S}{\faUnlock}\hspace{-1mm}
		}\right\}\textnormal{.}
	\end{align*}
	Finally, we define
	\begin{align*}
		\textnormal{generality}_\textnormal{\itshape algorithm} = \begin{cases}
		\textnormal{0} & \textnormal{if {\itshape algorithm} is a general purpose algorithm} \\
		\textnormal{2} & \textnormal{else}
		\end{cases}
	\end{align*}
	to take into account if the algorithm is suitable for general use.\\

	\textcolor{themeaccentsecondary}{TBD: In the algorithm scores given in ID cards, we currently use \textnormal{impl\textsubscript{\itshape algorithm}} = 0 in the respective computations because the necessary values for implementation complexity and size are still \tbd. This will be corrected later.}
\end{algorithmbox}
\vspace{-2mm}
\begin{algorithmbox}{Example: ML-DSA Overall Usability Score}
	\tiny
	We calculate
	\vspace{-4mm}
	\begin{align*}
		\textnormal{score\textsubscript{\itshape ML-DSA-44}} &= \textnormal{10} - \avg\left\{{
		\tripleicon{\montserratbold 3}{\faMicrochip}{themeyellow}{0.6}{\faKey}\hspace{-1mm}\textnormal{, } %
		\tripleicon{\montserratbold 3}{\faMicrochip}{themeyellow}{0.6}{\faPen}\hspace{-1mm}\textnormal{, } %
		\tripleicon{\montserratbold 2}{\faMicrochip}{themegreen}{0.6}{\faQuestionCircle}\hspace{-1mm}\textnormal{, } %
		\doubleicon{\montserratbold 1}{\faPen}{themegreen}{0.6}\hspace{-1mm}\textnormal{, } %
		\doubleicon{\montserratbold 5}{\faKey}{themeorange}{0.6}\hspace{-1mm}
		}\right\} = \textnormal{10} - \textnormal{2.8} = \textnormal{7.2}
	\end{align*}
	Similarly, we obtain \textnormal{score\textsubscript{\itshape ML-DSA-65}} = 7.0 and \textnormal{score\textsubscript{\itshape ML-DSA-87}} = 6.2. Furthermore, $\textnormal{categories}_\textnormal{\itshape ML-DSA}$ = 3 since ML-DSA offers the three security categories \doubleicon{\montserratbold II}{\faSun[regular]}{themered}{0.6}\hspace{-2mm}, \doubleicon{\montserratbold III}{\faSun[regular]}{themeorange}{0.6}\hspace{-2mm}, and \doubleicon{\montserratbold V}{\faSun[regular]}{themegreen}{0.6}\hspace{-2mm}, and $\textnormal{generality}_\textnormal{\itshape ML-DSA}$ = 0 since ML-DSA is a general purpose signature algorithm. This results in an overall usability score of 6.55:\\[\baselineskip]

	\begin{minipage}[T]{0.25\textwidth}
		\GreenAbsoluteSpeedometer(0,0)[2.25cm][0.75]{Unusable/1-3,Poor/4-5,Good/6-8,Perfect/9-10}{ML-DSA Overall Usability Score}{6.55}{10}{\tiny}{\tiny \montserratsemibold}
	\end{minipage}
	\hfill
	\begin{minipage}[T]{0.75\textwidth}
	\vspace{-\baselineskip}
		\begin{align*}
			\textnormal{score\textsubscript{\itshape ML-DSA}} &= \max\left\{\textnormal{0, }\avg\left\{\textnormal{score}_\textnormal{\itshape ML-DSA-44}\textnormal{, score}_\textnormal{\itshape ML-DSA-65}\textnormal{, score}_\textnormal{\itshape ML-DSA-87}\right\} - \frac{\textnormal{1}}{\textnormal{8}}\cdot\left(\textnormal{5}-\textnormal{categories}_\textnormal{\itshape ML-DSA}\right) - \frac{\textnormal{1}}{\textnormal{4}}\cdot\textnormal{impl\textsubscript{\itshape ML-DSA}}- \textnormal{generality}_\textnormal{\itshape ML-DSA}\right\}\\
			&=\max\left\{\textnormal{0, }\avg\left\{\textnormal{7.2, 7.0, 6.2}\right\} - \frac{\textnormal{1}}{\textnormal{8}}\cdot\left(\textnormal{5-3}\right) - \textnormal{0} - \textnormal{0}\right\}\\
			&=\max\left\{\textnormal{0, }\textnormal{6.8} - \textnormal{0.25}\right\}\\
			&=\max\left\{\textnormal{0, }\textnormal{6.55}\right\}\\
			&=\textnormal{6.55}
		\end{align*}
	\textcolor{themeaccentsecondary}{TBD: We use \textnormal{impl\textsubscript{\itshape ML-DSA}} = 0 because the necessary values to compute it are still \tbd, cf. ML-DSA ID card. This will be corrected later.}\\
	\end{minipage}
\end{algorithmbox}
