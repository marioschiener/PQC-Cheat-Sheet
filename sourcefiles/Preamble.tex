% !TeX root = ../PQC_Cheat_Sheet.tex

% Author: Mario Schiener, 2024
% LinkedIn: https://www.linkedin.com/in/mario-schiener/
% Github: https://github.com/marioschiener

% TODO This may include a few more imports than necessary and could be cleaned up a bit. That's just what happens when re-using prembles from other templates and documents.

\usepackage[utf8]{inputenc}
\usepackage[english,ngerman]{babel}
\usepackage{tgadventor} % font
\usepackage{fontspec}% requires XeLaTex or LuaLaTeX
\usepackage{fontawesome5}
\usepackage[table, dvipsnames]{xcolor}
\usepackage{tcolorbox}
\usepackage{svg}
\usepackage{wrapfig}
\usepackage{booktabs}
\usepackage{rotating}
\usepackage{graphicx}
\usepackage{times}
\usepackage{odsfile}
\usepackage{smartdiagram}
\usepackage{tikz}
\usetikzlibrary{matrix}
\usetikzlibrary{calc}
\usetikzlibrary{decorations.text,shadings, backgrounds, shapes.misc, positioning, arrows.meta, chains, shapes.symbols}	%for rounded rectangle ``buttons'', Speedometer diagram
\usepackage{siunitx}
\usepackage{tabularray}
\usepackage{tabularx}
\usepackage{multirow}
\usepackage{adjustbox}
\usepackage{enumitem}
\usepackage{makecell}
\usepackage{xifthen}
\usepackage{tikz}
\usepackage{lipsum}
\usepackage{float}
\usepackage{amssymb}
\usepackage{amsmath}
\usepackage{hyperref}
\usepackage[toc, acronym]{glossaries} % keep after hyperref
\usepackage{scrlayer-scrpage}
\usepackage[a4paper]{geometry}

% ==========================================================================================================
% COLORS, FONTS, LINKS
% ==========================================================================================================

% Colors
\definecolor{themeaccent}{RGB}{251,178,10}
\definecolor{themeblue}{RGB}{9,102,242}
\definecolor{themegreen}{RGB}{7,244,50}
\definecolor{themered}{RGB}{244,14,56}
\definecolor{themeyellow}{RGB}{255,204,7}
\definecolor{themeorange}{RGB}{251,151,10}
\definecolor{themewhite}{RGB}{220,220,220}	% Black!50
\definecolor{themegreylight}{RGB}{128,128,128}	% Black!50
\definecolor{themegreymedium}{RGB}{68,70,74}
\definecolor{themegreydark}{RGB}{64,64,64}	% Black!25
\definecolor{themegreyverydark}{RGB}{32,32,32}

% Fonts
\newfontfamily\montserrat{Montserrat-Alt1} % requires XeLaTex or LuaLaTeX
\newfontfamily\montserratlight{Montserrat-Alt1-ExtraLight} % requires XeLaTex or LuaLaTeX
\newfontfamily\montserratthin{Montserrat-Alt1-Thin} % requires XeLaTex or LuaLaTeX
\newfontfamily\montserratmedium{Montserrat-Alt1-Medium} % requires XeLaTex or LuaLaTeX
\newfontfamily\montserratsemibold{Montserrat-Alt1-SemiBold} % requires XeLaTex or LuaLaTeX
\newfontfamily\montserratbold{Montserrat-Alt1-Bold} % requires XeLaTex or LuaLaTeX

\setsansfont[Ligatures=TeX]{Tex Gyre Adventor}
\setmainfont[Ligatures=TeX]{Tex Gyre Adventor}
\renewcommand{\familydefault}{\sfdefault}


% TODO is this currently used in this document?
\newcommand{\highlight}[1]{\textit{\textcolor{themeaccent}{#1}}}

\parindent=0pt

% Links
\hypersetup{
    colorlinks=true,
    pdfborderstyle={/S/U/W 1},
    linkcolor=black,
    linkbordercolor=themeyellow,
    urlcolor=themeblue,
    urlbordercolor=themeaccent,
    citecolor=themegreylight
    }



\newcommand{\email}[2][]{\ifthenelse{\equal{#1}{}}{\href{mailto:#2}{\texttt{#2}}}{\href{mailto:#2}{\texttt{#1\,\,#2}}}}
\newcommand{\iconurl}[2][]{\ifthenelse{\equal{#1}{}}{\url{#2}}{\textcolor{themeaccent}{#1}\,\,\url{#2}}}
\newcommand{\iconhref}[3][]{\ifthenelse{\equal{#1}{}}{\href{#2}{#3}}{\href{#2}{#1 #3}}}

\newcommand{\tbd}{\textcolor{themeaccentsecondary}{TBD}}
\newcommand{\tbv}{\textcolor{themeaccentsecondary}{TBV}}

\DeclareMathOperator{\avg}{avg}


% ==========================================================================================================
% Vertically centered inline images
% ==========================================================================================================

\newcommand{\vcenteredincludesvg}[2]{\begingroup\setbox0=\hbox{\includesvg[#2]{#1}}\parbox{\wd0}{\box0}\endgroup}
\newcommand{\vcentered}[1]{\begingroup\setbox0=\hbox{#1}\parbox{\wd0}{\box0}\endgroup}


% ==========================================================================================================
% PAGE THEME& GEOMETRY
% ==========================================================================================================

\newcommand{\setpagetheme}[3]{

	\pagestyle{scrheadings}
	\clearpairofpagestyles
	%\clearscrheadfoot
	\automark[subsection]{section}


	% Margins and Background
	\newgeometry{left=1cm,right=1cm,top=2.4cm,bottom=0.75cm,headsep=0.75cm, footskip=2cm,includefoot}


	% Picture
	\AddToHook{shipout/background}{
		\begin{tikzpicture}[remember picture,overlay]
   				 \node[opacity=1] at (current page.center) {\includesvg{resources/Header}};
		\end{tikzpicture}
	}

	\addtokomafont{pagenumber}{\scshape\color{themegreylight}}

	%Header
	\ihead{\Huge \montserrat \textcolor{themeaccent}{#1}}

	%Footer
	\ofoot{\textcolor{themegreylight}{\pagemark}}
	\ifoot{\normalfont\scshape \textcolor{themegreylight}{#2}}
	\cfoot{\normalfont\scshape \textcolor{themegreylight}{#3}\\\textcolor{themegreylight}{\tiny No responsibility is taken for the correctness of this information.}}
}




% ==========================================================================================================
% ICONS
% ==========================================================================================================

% Note that for most icons, we will use some negative \hspace{- xyz} to shift folloing text closer to the icon. This is not done in the definition
% of the icons here on purpose as the right amount of shifting needed for nice looking results depends on the font size used in surrounding text or tabular environments.
% So, we do this right where we use the icons, even if it makes the code a bit ugly admittedly.

\newcommand{\icon}[4]
		\begin{tikzpicture}[scale=#4, transform shape]%
		\draw[draw=none,fill=#2] (0,0) circle (0.5);%
		\node[color=#3](icon) at (0,0) {#1};%
		\end{tikzpicture}
	}
}


\newcommand{\doubleicon}[5][themegreydark]
	\begin{tikzpicture}[scale=#5, transform shape]%
		\draw[draw=none,fill=#4] (0,0) circle (0.4);%
		\draw[draw=none,fill=themegreydark] (0.35,-0.2) circle (0.19);%
		\node[color=#1](icon) at (0,0) {\normalsize #2};%
		\node[color=themewhite,scale=0.9](icon) at (0.35,-0.2) {#3};%
	\end{tikzpicture}
	}
}

\newcommand{\tripleicon}[6][themegreydark]
	\begin{tikzpicture}[scale=#5, transform shape]%
		\draw[draw=none,fill=#4] (0,0) circle (0.4);%
		\draw[draw=none,fill=themegreydark] (0.35,-0.2) circle (0.19);%
		\draw[draw=none,fill=themegreydark] (0.35,0.2) circle (0.19);%
		\node[color=#1](icon) at (0,0) {\normalsize #2};%
		\node[color=themewhite](icon) at (0.35,-0.2) {#3};%
		\node[color=themewhite,scale=0.9](icon) at (0.35,0.2) {#6};%
	\end{tikzpicture}
	}
}

\newcommand{\tripleicongeneric}[6][themegreydark]
	\begin{tikzpicture}[scale=#5, transform shape]%
		\draw[draw=none,fill=#4] (0,0) circle (0.4);%
		\draw[draw=none,fill=themegreydark] (0.35,-0.2) circle (0.19);%
		\draw[color=themegreydark] (0.35,0.2) circle (0.19);%
		\node[color=#1](icon) at (0,0) {\normalsize #2};%
		\node[color=themewhite](icon) at (0.35,-0.2) {#3};%
		\node[color=themegreydark,scale=0.9](icon) at (0.35,0.2) {#6};%
	\end{tikzpicture}
	}
}

\newcommand{\lettericon}[4]
		\begin{tikzpicture}[scale=#4, transform shape]%
		\draw[draw=none,fill=#2] (-0.1,0) circle (0.125);%
		\node[color=#1](icon) at (-0.1,0) {\bfseries #3};%
		\end{tikzpicture}
	}
}


% ==========================================================================================================
% RANKINGS DIAGRAM
% ==========================================================================================================

\newcommand{\ranking}[4]{%
	\scalebox{#4}{
		\smartdiagramset{%
		set color list={themegreen,themeyellow,themeorange,themered,themered!75!black},
		sequence item border color=themegreydark,
		sequence item font size=\footnotesize,
		sequence item text color=themegreydark,
		sequence item width=2cm,
		sequence item height=1cm
		}
		#1\vcentered{\smartdiagram[sequence diagram]{#2}}#3
	}
}


% ==========================================================================================================
% BOXES
% ==========================================================================================================


\newenvironment{algorithmbox}[1]{%
		\begin{tcolorbox}[colback=themewhite,colframe=themegreydark,title=#1, coltitle=themeaccent, fonttitle=\scshape, coltext=themegreydark, halign=left]
	}%
	{%
	\end{tcolorbox}
}


% ==========================================================================================================
% Info Graphics
% ==========================================================================================================

% TODO maybe delete later, currently not used

\newlength{\heightsymbols}
\newlength{\heightbutton}
\newlength{\heighttoraise}

% Scale
% format: percentage, total number of symbols, symbol, font formatting, color
\newcommand{\absolutescale}[7][]{%
   	\pgfmathmultiply{#2}{#3}%
	\pgfmathdivide{\pgfmathresult}{100}%
	\pgfmathsetmacro{\bars}{int(round(\pgfmathresult))}
	\pgfmathadd{\pgfmathresult}{1}
	\pgfmathsetmacro{\barsnext}{int(round(\pgfmathresult)}


	%Some measurements for vertically centering everything
	\settoheight{\heightsymbols}{\hbox{#4}}
	\settoheight{\heightbutton}{\hbox{\custombutton{#5 30}{white}{white}}}
	\setlength{\heighttoraise}{\dimexpr(\heightbutton - \heightsymbols)/2\relax}

	{
	#5 % size
	\noindent\raisebox{\heighttoraise}{
		\foreach \j in {1,...,#3}{
			\ifnum\numexpr\j-1<#2\relax
				\textcolor{#6}{#4}\ignorespaces
			\else
				\textcolor{#7}{#4}\ignorespaces
			\fi
		}
	}\ifthenelse{\isempty{#1}}
	{}
	{\ignorespaces\,\,\custombutton{#5 #2/#3}{#6}{#7}}
	}
}


% Button
% format: content, fill color, text color

\newcommand{\custombutton}[3]{%
	\vcentered{%
		\begin{tikzpicture}
			\node [rounded rectangle, minimum width=1.25cm, minimum height=4mm, inner sep=2pt, fill=#2, text=#3, baseline=-0.5ex] {#1};
		\end{tikzpicture}
		}
}



% Speedometers

\pgfdeclarehorizontalshading{grad}{50bp}{
    color(0cm)=(themered);
	color(20bp)=(themeorange);
    color(30bp)=(themeyellow);
    color(50bp)=(themegreen)
}


% TODO maybe delete later, currently not used (only green variant below)

\def\RedAbsoluteSpeedometer(#1)[#2][#3]#4#5#6#7#8#9{%
	\begin{tikzpicture}
		\begin{scope}[shift={(#1)}]
			\edef\Mycount{0}%Variable to get the number of entries in the array
			\ifthenelse{\isempty{#4}}{\xdef\Mycount{#7}}{
				\foreach \element in {#4}{%For each element
					\pgfmathparse{int(\Mycount+1)}%increase in one unit the last value of \Mycount
					\xdef\Mycount{\pgfmathresult}%Refresh the  \Mycount value
				}%\Mycount get the value of the number of elements in given by #4 array input.
			}
			\pgfmathparse{180/\Mycount)}%Using the value of the number of inputs in the array
			\edef\Angle{\pgfmathresult}%Get the size of the angle sector for some range of valuis in the Gauje
			\pgfmathparse{180-(#6/#7)*180}
			\edef\ValueAngle{\pgfmathresult}


			% full half circle, goes in the background
			\def\leftpath{(180:#2)
					arc (180:0:#2)
					-- (0:#2*0.55)
					arc (0:180:#2*0.55)
					-- cycle;}
			\draw[draw=none,thick]\leftpath


			% segment to the right (> valueangle)
			\def\rightpath{(\ValueAngle:#2)
					arc (\ValueAngle:0:#2)
					-- (0:#2*0.55)
					arc (0:\ValueAngle:#2*0.55)
					-- cycle;}
			%\draw[draw=none,thick]\rightpath


			% full half circle with complete color scale
			\begin{scope}[on background layer]
				\shade[
					shading=grad,
					shading angle=180
					]\leftpath
			\end{scope}


			% cover the unused part through above layer
			\begin{scope}[on above layer]
				\shade[
					bottom color=themegreydark,
					top color=themegreydark,
					]\rightpath;
			\end{scope}


			\foreach \Stext/\Mtext [count=\i from 0] in {#4}{
				\def\Mypath{(180-\Angle*\i:#2)
					arc (180-\Angle*\i:180-\Angle*\i-\Angle:#2)
					-- (180-\Angle*\i-\Angle:#2*0.55)
					arc (180-\Angle*\i-\Angle:180-\Angle*\i:#2*0.55)
					-- cycle;}
				\draw[draw=none,thick]\Mypath

				% Numbers and text
				\path[decoration={text along path,text={| #8 |\Stext},text align=center},decorate]
					(180-\Angle*\i:#2*1.05) arc (180-\Angle*\i:180-\Angle*\i-\Angle:#2*1.05);


				\path[decoration={text along path,text={| #9 |\Mtext},text align=center, text color=white},decorate]
					(180-\Angle*\i:#2*#3) arc (180-\Angle*\i:180-\Angle*\i-\Angle:#2*#3);

			}

			\pgfmathparse{180-((180-\Angle)/#7*#6)}
			%\xdef\Value{\pgfmathresult} % by Nice cfr observation
			%\fill[themegreylight,rotate=\pgfmathresult] (90:#2*0.03)--(0:#2*0.75) -- (-90:#2*0.03)--cycle;
			%\fill [themegreylight](0,0) circle (#2*0.08);
			%\fill [themegreydark](0,0) circle (#2*0.03);

			% Caption
			\draw(-90:#2*0.15) node[font=\bfseries] {#5};
			\draw(-90:#2*-0.15) node[font=\bfseries] {\large \custombutton{#6}{themegreydark}{themewhite}};

		\end{scope}
	\end{tikzpicture}
}




\def\GreenAbsoluteSpeedometer(#1)[#2][#3]#4#5#6#7#8#9{%
	\begin{tikzpicture}
		\begin{scope}[shift={(#1)}]
			\edef\Mycount{0}%Variable to get the number of entries in the array
			\ifthenelse{\isempty{#4}}{\xdef\Mycount{#7}}{
				\foreach \element in {#4}{%For each element
					\pgfmathparse{int(\Mycount+1)}%increase in one unit the last value of \Mycount
					\xdef\Mycount{\pgfmathresult}%Refresh the  \Mycount value
				}%\Mycount get the value of the number of elements in given by #4 array input.
			}
			\pgfmathparse{180/\Mycount)}%Using the value of the number of inputs in the array
			\edef\Angle{\pgfmathresult}%Get the size of the angle sector for some range of valuis in the Gauje
			\pgfmathparse{180-(#6/#7)*180}
			\edef\ValueAngle{\pgfmathresult}


			% full half circle, goes in the background
			\def\leftpath{(180:#2)
					arc (180:0:#2)
					-- (0:#2*0.55)
					arc (0:180:#2*0.55)
					-- cycle;}
			\draw[draw=none,thick]\leftpath


			% segment to the right (> valueangle)
			\def\rightpath{(\ValueAngle:#2)
					arc (\ValueAngle:0:#2)
					-- (0:#2*0.55)
					arc (0:\ValueAngle:#2*0.55)
					-- cycle;}
			%\draw[draw=none,thick]\rightpath


			% full half circle with complete color scale
			\begin{scope}[on background layer]
				\shade[
					shading=grad
					]\leftpath
			\end{scope}


			% cover the unused part through above layer
			\begin{scope}[on above layer]
				\shade[
					bottom color=themegreydark,
					top color=themegreydark,
					]\rightpath;
			\end{scope}


			\foreach \Stext/\Mtext [count=\i from 0] in {#4}{
				\def\Mypath{(180-\Angle*\i:#2)
					arc (180-\Angle*\i:180-\Angle*\i-\Angle:#2)
					-- (180-\Angle*\i-\Angle:#2*0.55)
					arc (180-\Angle*\i-\Angle:180-\Angle*\i:#2*0.55)
					-- cycle;}
				\draw[draw=none,thick]\Mypath

				% Numbers and text
				\path[decoration={text along path,text={| #8 |\Stext},text align=center},decorate]
					(180-\Angle*\i:#2*1.05) arc (180-\Angle*\i:180-\Angle*\i-\Angle:#2*1.05);


				\path[decoration={text along path,text={| #9 |\Mtext},text align=center, text color=white},decorate]
					(180-\Angle*\i:#2*#3) arc (180-\Angle*\i:180-\Angle*\i-\Angle:#2*#3);

			}

			\pgfmathparse{180-((180-\Angle)/#7*#6)}
			%\xdef\Value{\pgfmathresult} % by Nice cfr observation
			%\fill[themegreylight,rotate=\pgfmathresult] (90:#2*0.03)--(0:#2*0.75) -- (-90:#2*0.03)--cycle;
			%\fill [themegreylight](0,0) circle (#2*0.08);
			%\fill [themegreydark](0,0) circle (#2*0.03);

			% Caption
			\draw(-90:#2*0.15) node[font=\bfseries] {#5};
			\draw(-90:#2*-0.15) node[font=\bfseries] {\large \custombutton{#6}{themegreydark}{themewhite}};

		\end{scope}
	\end{tikzpicture}
}


